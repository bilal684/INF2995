本教程简介文件对象模型(\+Document Object Model, D\+O\+M)\+A\+P\+I。

如用法一览中所示,可以解析一个\+J\+S\+O\+N至\+D\+O\+M,然后就可以轻松查询及修改\+D\+O\+M,并最终转换回\+J\+S\+O\+N。\hypertarget{md_Commun_Externe_RapidJSON_doc_tutorial.zh-cn_ValueDocument}{}\section{Value \& Document}\label{md_Commun_Externe_RapidJSON_doc_tutorial.zh-cn_ValueDocument}
每个\+J\+S\+O\+N值都储存为{\ttfamily Value}类,而{\ttfamily Document}类则表示整个\+D\+O\+M,它存储了一个\+D\+O\+M树的根{\ttfamily Value}。\+Rapid\+J\+S\+O\+N的所有公开类型及函数都在{\ttfamily rapidjson}命名空间中。\hypertarget{md_Commun_Externe_RapidJSON_doc_tutorial.zh-cn_QueryValue}{}\section{Query Value}\label{md_Commun_Externe_RapidJSON_doc_tutorial.zh-cn_QueryValue}
在本节中,我们会使用到{\ttfamily example/tutorial/tutorial.\+cpp}中的代码片段。

假设我们用\+C语言的字符串储存一个\+J\+S\+O\+N({\ttfamily const char$\ast$ json}): 
\begin{DoxyCode}
\{
    \textcolor{stringliteral}{"hello"}: \textcolor{stringliteral}{"world"},
    \textcolor{stringliteral}{"t"}: true ,
    \textcolor{stringliteral}{"f"}: \textcolor{keyword}{false},
    \textcolor{stringliteral}{"n"}: null,
    \textcolor{stringliteral}{"i"}: 123,
    \textcolor{stringliteral}{"pi"}: 3.1416,
    \textcolor{stringliteral}{"a"}: [1, 2, 3, 4]
\}
\end{DoxyCode}


把它解析至一个{\ttfamily Document}: 
\begin{DoxyCode}
\textcolor{preprocessor}{#include "\hyperlink{document_8h}{rapidjson/document.h}"}

\textcolor{keyword}{using namespace }\hyperlink{namespacerapidjson}{rapidjson};

\textcolor{comment}{// ...}
\hyperlink{class_generic_document}{Document} document;
document.\hyperlink{class_generic_document_aebd4e7fddd80c1e1174837aee6d2159b}{Parse}(json);
\end{DoxyCode}


那么现在该\+J\+S\+O\+N就会被解析至{\ttfamily document}中,成为一棵$\ast$\+D\+O\+M树$\ast$\+:



自从\+R\+FC 7159作出更新,合法\+J\+S\+O\+N文件的根可以是任何类型的\+J\+S\+O\+N值。而在较早的\+R\+FC 4627中,根值只允许是\+Object或\+Array。而在上述例子中,根是一个\+Object。 
\begin{DoxyCode}
assert(document.IsObject());
\end{DoxyCode}


让我们查询一下根\+Object中有没有{\ttfamily \char`\"{}hello\char`\"{}}成员。由于一个{\ttfamily Value}可包含不同类型的值,我们可能需要验证它的类型,并使用合适的\+A\+P\+I去获取其值。在此例中,{\ttfamily \char`\"{}hello\char`\"{}}成员关联到一个\+J\+S\+ON String。 
\begin{DoxyCode}
assert(document.HasMember(\textcolor{stringliteral}{"hello"}));
assert(document[\textcolor{stringliteral}{"hello"}].IsString());
printf(\textcolor{stringliteral}{"hello = %s\(\backslash\)n"}, document[\textcolor{stringliteral}{"hello"}].GetString());
\end{DoxyCode}



\begin{DoxyCode}
1 world
\end{DoxyCode}


J\+S\+ON True/\+False值是以{\ttfamily bool}表示的。 
\begin{DoxyCode}
assert(document[\textcolor{stringliteral}{"t"}].IsBool());
printf(\textcolor{stringliteral}{"t = %s\(\backslash\)n"}, document[\textcolor{stringliteral}{"t"}].GetBool() ? \textcolor{stringliteral}{"true"} : \textcolor{stringliteral}{"false"});
\end{DoxyCode}



\begin{DoxyCode}
1 true
\end{DoxyCode}


J\+S\+ON Null值可用{\ttfamily Is\+Null()}查询。 
\begin{DoxyCode}
printf(\textcolor{stringliteral}{"n = %s\(\backslash\)n"}, document[\textcolor{stringliteral}{"n"}].IsNull() ? \textcolor{stringliteral}{"null"} : \textcolor{stringliteral}{"?"});
\end{DoxyCode}



\begin{DoxyCode}
1 null
\end{DoxyCode}


J\+S\+ON Number类型表示所有数值。然而,\+C++需要使用更专门的类型。


\begin{DoxyCode}
assert(document[\textcolor{stringliteral}{"i"}].IsNumber());

\textcolor{comment}{// 在此情况下,IsUint()/IsInt64()/IsUInt64()也会返回 true}
assert(document[\textcolor{stringliteral}{"i"}].IsInt());          
printf(\textcolor{stringliteral}{"i = %d\(\backslash\)n"}, document[\textcolor{stringliteral}{"i"}].GetInt());
\textcolor{comment}{// 另一种用法: (int)document["i"]}

assert(document[\textcolor{stringliteral}{"pi"}].IsNumber());
assert(document[\textcolor{stringliteral}{"pi"}].IsDouble());
printf(\textcolor{stringliteral}{"pi = %g\(\backslash\)n"}, document[\textcolor{stringliteral}{"pi"}].GetDouble());
\end{DoxyCode}



\begin{DoxyCode}
1 i = 123
2 pi = 3.1416
\end{DoxyCode}


J\+S\+ON Array包含一些元素。 
\begin{DoxyCode}
\textcolor{comment}{// 使用引用来连续访问,方便之余还更高效。}
\textcolor{keyword}{const} \hyperlink{class_generic_value}{Value}& a = document[\textcolor{stringliteral}{"a"}];
assert(a.IsArray());
\textcolor{keywordflow}{for} (\hyperlink{rapidjson_8h_a5ed6e6e67250fadbd041127e6386dcb5}{SizeType} i = 0; i < a.Size(); i++) \textcolor{comment}{// 使用 SizeType 而不是 size\_t}
        printf(\textcolor{stringliteral}{"a[%d] = %d\(\backslash\)n"}, i, a[i].GetInt());
\end{DoxyCode}



\begin{DoxyCode}
1 a[0] = 1
2 a[1] = 2
3 a[2] = 3
4 a[3] = 4
\end{DoxyCode}


注意,\+Rapid\+J\+S\+O\+N并不自动转换各种\+J\+S\+O\+N类型。例如,对一个\+String的\+Value调用{\ttfamily Get\+Int()}是非法的。在调试模式下,它会被断言失败。在发布模式下,其行为是未定义的。

以下将会讨论有关查询各类型的细节。\hypertarget{md_Commun_Externe_RapidJSON_doc_tutorial.zh-cn_QueryArray}{}\subsection{Query Array}\label{md_Commun_Externe_RapidJSON_doc_tutorial.zh-cn_QueryArray}
缺省情况下,{\ttfamily Size\+Type}是{\ttfamily unsigned}的typedef。在多数系统中,\+Array最多能存储2$^\wedge$32-\/1个元素。

你可以用整数字面量访问元素,如{\ttfamily a\mbox{[}0\mbox{]}}、{\ttfamily a\mbox{[}1\mbox{]}}、{\ttfamily a\mbox{[}2\mbox{]}}。

Array与{\ttfamily std\+::vector}相似,除了使用索引,也可使用迭待器来访问所有元素。 
\begin{DoxyCode}
\textcolor{keywordflow}{for} (\hyperlink{class_generic_value}{Value::ConstValueIterator} itr = a.Begin(); itr != a.End(); ++itr)
    printf(\textcolor{stringliteral}{"%d "}, itr->GetInt());
\end{DoxyCode}


还有一些熟悉的查询函数:
\begin{DoxyItemize}
\item {\ttfamily Size\+Type Capacity() const}
\item {\ttfamily bool Empty() const}
\end{DoxyItemize}\hypertarget{md_Commun_Externe_RapidJSON_doc_tutorial.zh-cn_QueryObject}{}\subsection{Query Object}\label{md_Commun_Externe_RapidJSON_doc_tutorial.zh-cn_QueryObject}
和\+Array相似,我们可以用迭代器去访问所有\+Object成员:


\begin{DoxyCode}
\textcolor{keyword}{static} \textcolor{keyword}{const} \textcolor{keywordtype}{char}* kTypeNames[] = 
    \{ \textcolor{stringliteral}{"Null"}, \textcolor{stringliteral}{"False"}, \textcolor{stringliteral}{"True"}, \textcolor{stringliteral}{"Object"}, \textcolor{stringliteral}{"Array"}, \textcolor{stringliteral}{"String"}, \textcolor{stringliteral}{"Number"} \};

\textcolor{keywordflow}{for} (\hyperlink{class_generic_member_iterator}{Value::ConstMemberIterator} itr = document.MemberBegin();
    itr != document.MemberEnd(); ++itr)
\{
    printf(\textcolor{stringliteral}{"Type of member %s is %s\(\backslash\)n"},
        itr->name.GetString(), kTypeNames[itr->value.GetType()]);
\}
\end{DoxyCode}



\begin{DoxyCode}
1 Type of member hello is String
2 Type of member t is True
3 Type of member f is False
4 Type of member n is Null
5 Type of member i is Number
6 Type of member pi is Number
7 Type of member a is Array
\end{DoxyCode}


注意,当{\ttfamily operator\mbox{[}$\,$\mbox{]}(const char$\ast$)}找不到成员,它会断言失败。

若我们不确定一个成员是否存在,便需要在调用{\ttfamily operator\mbox{[}$\,$\mbox{]}(const char$\ast$)}前先调用{\ttfamily Has\+Member()}。然而,这会导致两次查找。更好的做法是调用{\ttfamily Find\+Member()},它能同时检查成员是否存在并返回它的\+Value:


\begin{DoxyCode}
\hyperlink{class_generic_member_iterator}{Value::ConstMemberIterator} itr = document.FindMember(\textcolor{stringliteral}{"hello"});
\textcolor{keywordflow}{if} (itr != document.MemberEnd())
    printf(\textcolor{stringliteral}{"%s %s\(\backslash\)n"}, itr->value.GetString());
\end{DoxyCode}
\hypertarget{md_Commun_Externe_RapidJSON_doc_tutorial.zh-cn_QueryNumber}{}\subsection{Querying Number}\label{md_Commun_Externe_RapidJSON_doc_tutorial.zh-cn_QueryNumber}
J\+S\+O\+N只提供一种数值类型──\+Number。数字可以是整数或实数。\+R\+FC 4627规定数字的范围由解析器指定。

由于\+C++提供多种整数及浮点数类型,\+D\+O\+M尝试尽量提供最广的范围及良好性能。

当解析一个\+Number时, 它会被存储在\+D\+O\+M之中,成为下列其中一个类型:

\tabulinesep=1mm
\begin{longtabu} spread 0pt [c]{*2{|X[-1]}|}
\hline
\rowcolor{\tableheadbgcolor}{\bf 类型 }&{\bf 描述  }\\\cline{1-2}
\endfirsthead
\hline
\endfoot
\hline
\rowcolor{\tableheadbgcolor}{\bf 类型 }&{\bf 描述  }\\\cline{1-2}
\endhead
{\ttfamily unsigned} &32位无号整数 \\\cline{1-2}
{\ttfamily int} &32位有号整数 \\\cline{1-2}
{\ttfamily uint64\+\_\+t} &64位无号整数 \\\cline{1-2}
{\ttfamily int64\+\_\+t} &64位有号整数 \\\cline{1-2}
{\ttfamily double} &64位双精度浮点数 \\\cline{1-2}
\end{longtabu}
当查询一个\+Number时, 你可以检查该数字是否能以目标类型来提取:

\tabulinesep=1mm
\begin{longtabu} spread 0pt [c]{*2{|X[-1]}|}
\hline
\rowcolor{\tableheadbgcolor}{\bf 查检 }&{\bf 提取  }\\\cline{1-2}
\endfirsthead
\hline
\endfoot
\hline
\rowcolor{\tableheadbgcolor}{\bf 查检 }&{\bf 提取  }\\\cline{1-2}
\endhead
{\ttfamily bool Is\+Number()} &不适用 \\\cline{1-2}
{\ttfamily bool Is\+Uint()} &{\ttfamily unsigned Get\+Uint()} \\\cline{1-2}
{\ttfamily bool Is\+Int()} &{\ttfamily int Get\+Int()} \\\cline{1-2}
{\ttfamily bool Is\+Uint64()} &{\ttfamily uint64\+\_\+t Get\+Uint64()} \\\cline{1-2}
{\ttfamily bool Is\+Int64()} &{\ttfamily int64\+\_\+t Get\+Int64()} \\\cline{1-2}
{\ttfamily bool Is\+Double()} &{\ttfamily double Get\+Double()} \\\cline{1-2}
\end{longtabu}
注意,一个整数可能用几种类型来提取,而无需转换。例如,一个名为{\ttfamily x}的\+Value包含123,那么{\ttfamily x.\+Is\+Int() == x.\+Is\+Uint() == x.\+Is\+Int64() == x.\+Is\+Uint64() == true}。但如果一个名为{\ttfamily y}的\+Value包含-\/3000000000,那么仅会令{\ttfamily x.\+Is\+Int64() == true}。

当要提取\+Number类型,{\ttfamily Get\+Double()}是会把内部整数的表示转换成{\ttfamily double}。注意{\ttfamily int} 和{\ttfamily unsigned}可以安全地转换至{\ttfamily double},但{\ttfamily int64\+\_\+t}及{\ttfamily uint64\+\_\+t}可能会丧失精度(因为{\ttfamily double}的尾数只有52位)。\hypertarget{md_Commun_Externe_RapidJSON_doc_tutorial.zh-cn_QueryString}{}\subsection{Query String}\label{md_Commun_Externe_RapidJSON_doc_tutorial.zh-cn_QueryString}
除了{\ttfamily Get\+String()},{\ttfamily Value}类也有一个{\ttfamily Get\+String\+Length()}。这里会解释个中原因。

根据\+R\+FC 4627,\+J\+S\+ON String可包含\+Unicode字符{\ttfamily U+0000},在\+J\+S\+O\+N中会表示为{\ttfamily \char`\"{}\textbackslash{}\textbackslash{}u0000\char`\"{}}。问题是,\+C/\+C++通常使用空字符结尾字符串(null-\/terminated string),这种字符串把`{\ttfamily \textbackslash{}0\textquotesingle{}}作为结束符号。

为了符合\+R\+FC 4627,\+Rapid\+J\+S\+O\+N支持包含{\ttfamily U+0000}的\+String。若你需要处理这些\+String,便可使用{\ttfamily Get\+String\+Length()}去获得正确的字符串长度。

例如,当解析以下的\+J\+S\+O\+N至{\ttfamily Document d}之后:


\begin{DoxyCode}
\{ \textcolor{stringliteral}{"s"} :  \textcolor{stringliteral}{"a\(\backslash\)u0000b"} \}
\end{DoxyCode}
 {\ttfamily \char`\"{}a\textbackslash{}\textbackslash{}u0000b\char`\"{}}值的正确长度应该是3。但{\ttfamily strlen()}会返回1。

{\ttfamily Get\+String\+Length()}也可以提高性能,因为用户可能需要调用{\ttfamily strlen()}去分配缓冲。

此外,{\ttfamily std\+::string}也支持这个构造函数:


\begin{DoxyCode}
string(\textcolor{keyword}{const} \textcolor{keywordtype}{char}* s, \textcolor{keywordtype}{size\_t} count);
\end{DoxyCode}


此构造函数接受字符串长度作为参数。它支持在字符串中存储空字符,也应该会有更好的性能。

\subsection*{比较两个\+Value}

你可使用{\ttfamily ==}及{\ttfamily !=}去比较两个\+Value。当且仅当两个\+Value的类型及内容相同,它们才当作相等。你也可以比较\+Value和它的原生类型值。以下是一个例子。


\begin{DoxyCode}
\textcolor{keywordflow}{if} (document[\textcolor{stringliteral}{"hello"}] == document[\textcolor{stringliteral}{"n"}]) \textcolor{comment}{/*...*/};    \textcolor{comment}{// 比较两个值}
\textcolor{keywordflow}{if} (document[\textcolor{stringliteral}{"hello"}] == \textcolor{stringliteral}{"world"}) \textcolor{comment}{/*...*/};          \textcolor{comment}{// 与字符串家面量作比较}
\textcolor{keywordflow}{if} (document[\textcolor{stringliteral}{"i"}] != 123) \textcolor{comment}{/*...*/};                  \textcolor{comment}{// 与整数作比较}
\textcolor{keywordflow}{if} (document[\textcolor{stringliteral}{"pi"}] != 3.14) \textcolor{comment}{/*...*/};                \textcolor{comment}{// 与double作比较}
\end{DoxyCode}


Array/\+Object顺序以它们的元素/成员作比较。当且仅当它们的整个子树相等,它们才当作相等。

注意,现时若一个\+Object含有重复命名的成员,它与任何\+Object作比较都总会返回{\ttfamily false}。\hypertarget{md_Commun_Externe_RapidJSON_doc_tutorial.zh-cn_CreateModifyValues}{}\section{Create/\+Modify Values}\label{md_Commun_Externe_RapidJSON_doc_tutorial.zh-cn_CreateModifyValues}
有多种方法去创建值。 当一个\+D\+O\+M树被创建或修改后,可使用{\ttfamily \hyperlink{class_writer}{Writer}}再次存储为\+J\+S\+O\+N。\hypertarget{md_Commun_Externe_RapidJSON_doc_tutorial.zh-cn_ChangeValueType}{}\subsection{Change Value Type}\label{md_Commun_Externe_RapidJSON_doc_tutorial.zh-cn_ChangeValueType}
当使用默认构造函数创建一个\+Value或\+Document,它的类型便会是\+Null。要改变其类型,需调用{\ttfamily Set\+X\+X\+X()}或赋值操作,例如:


\begin{DoxyCode}
\hyperlink{class_generic_document}{Document} d; \textcolor{comment}{// Null}
d.SetObject();

\hyperlink{class_generic_value}{Value} v;    \textcolor{comment}{// Null}
v.SetInt(10);
v = 10;     \textcolor{comment}{// 简写,和上面的相同}
\end{DoxyCode}


\subsubsection*{构造函数的各个重载}

几个类型也有重载构造函数:


\begin{DoxyCode}
\hyperlink{class_generic_value}{Value} b(\textcolor{keyword}{true});    \textcolor{comment}{// 调用Value(bool)}
\hyperlink{class_generic_value}{Value} i(-123);    \textcolor{comment}{// 调用 Value(int)}
\hyperlink{class_generic_value}{Value} u(123u);    \textcolor{comment}{// 调用Value(unsigned)}
\hyperlink{class_generic_value}{Value} d(1.5);     \textcolor{comment}{// 调用Value(double)}
\end{DoxyCode}


要重建空\+Object或\+Array,可在默认构造函数后使用 {\ttfamily Set\+Object()}/{\ttfamily Set\+Array()},或一次性使用{\ttfamily Value(\+Type)}:


\begin{DoxyCode}
\hyperlink{class_generic_value}{Value} o(\hyperlink{rapidjson_8h_a1d1cfd8ffb84e947f82999c682b666a7a146f46700e905e8df96a6a90b5c7640f}{kObjectType});
\hyperlink{class_generic_value}{Value} a(\hyperlink{rapidjson_8h_a1d1cfd8ffb84e947f82999c682b666a7af41527d6925efa3c5c3dadb23dfef7c8}{kArrayType});
\end{DoxyCode}
\hypertarget{md_Commun_Externe_RapidJSON_doc_tutorial.zh-cn_MoveSemantics}{}\subsection{Move Semantics}\label{md_Commun_Externe_RapidJSON_doc_tutorial.zh-cn_MoveSemantics}
在设计\+Rapid\+J\+S\+O\+N时有一个非常特别的决定,就是\+Value赋值并不是把来源\+Value复制至目的\+Value,而是把把来源\+Value转移(move)至目的\+Value。例如:


\begin{DoxyCode}
\hyperlink{class_generic_value}{Value} a(123);
\hyperlink{class_generic_value}{Value} b(456);
b = a;         \textcolor{comment}{// a变成Null,b变成数字123。}
\end{DoxyCode}




为什么?此语意有何优点?

最简单的答案就是性能。对于固定大小的\+J\+S\+O\+N类型(\+Number、\+True、\+False、\+Null),复制它们是简单快捷。然而,对于可变大小的\+J\+S\+O\+N类型(\+String、\+Array、\+Object),复制它们会产生大量开销,而且这些开销常常不被察觉。尤其是当我们需要创建临时\+Object,把它复制至另一变量,然后再析构它。

例如,若使用正常$\ast$复制$\ast$语意:


\begin{DoxyCode}
\hyperlink{class_generic_value}{Value} o(\hyperlink{rapidjson_8h_a1d1cfd8ffb84e947f82999c682b666a7a146f46700e905e8df96a6a90b5c7640f}{kObjectType});
\{
    \hyperlink{class_generic_value}{Value} contacts(\hyperlink{rapidjson_8h_a1d1cfd8ffb84e947f82999c682b666a7af41527d6925efa3c5c3dadb23dfef7c8}{kArrayType});
    \textcolor{comment}{// 把元素加进contacts数组。}
    \textcolor{comment}{// ...}
    o.AddMember(\textcolor{stringliteral}{"contacts"}, contacts, d.\hyperlink{class_generic_document_aa4609d6b19f86aec1a6b96edf2c27686}{GetAllocator}());  \textcolor{comment}{// 深度复制contacts (可能有大量内存分配)}
    \textcolor{comment}{// 析构contacts。}
\}
\end{DoxyCode}




那个{\ttfamily o} Object需要分配一个和contacts相同大小的缓冲区,对conacts做深度复制,并最终要析构contacts。这样会产生大量无必要的内存分配/释放,以及内存复制。

有一些方案可避免实质地复制这些数据,例如引用计数(reference counting)、垃圾回收(garbage collection, G\+C)。

为了使\+Rapid\+J\+S\+O\+N简单及快速,我们选择了对赋值采用$\ast$转移$\ast$语意。这方法与{\ttfamily std\+::auto\+\_\+ptr}相似,都是在赋值时转移拥有权。转移快得多简单得多,只需要析构原来的\+Value,把来源{\ttfamily memcpy()}至目标,最后把来源设置为\+Null类型。

因此,使用转移语意后,上面的例子变成:


\begin{DoxyCode}
\hyperlink{class_generic_value}{Value} o(\hyperlink{rapidjson_8h_a1d1cfd8ffb84e947f82999c682b666a7a146f46700e905e8df96a6a90b5c7640f}{kObjectType});
\{
    \hyperlink{class_generic_value}{Value} contacts(\hyperlink{rapidjson_8h_a1d1cfd8ffb84e947f82999c682b666a7af41527d6925efa3c5c3dadb23dfef7c8}{kArrayType});
    \textcolor{comment}{// adding elements to contacts array.}
    o.AddMember(\textcolor{stringliteral}{"contacts"}, contacts, d.\hyperlink{class_generic_document_aa4609d6b19f86aec1a6b96edf2c27686}{GetAllocator}());  \textcolor{comment}{// 只需 memcpy()
       contacts本身至新成员的Value(16字节)}
    \textcolor{comment}{// contacts在这里变成Null。它的析构是平凡的。}
\}
\end{DoxyCode}




在\+C++11中这称为转移赋值操作(move assignment operator)。由于\+Rapid\+J\+S\+ON 支持\+C++03,它在赋值操作采用转移语意,其它修改形函数如{\ttfamily Add\+Member()}, {\ttfamily Push\+Back()}也采用转移语意。\hypertarget{md_Commun_Externe_RapidJSON_doc_tutorial.zh-cn_TemporaryValues}{}\subsubsection{Move semantics and temporary values}\label{md_Commun_Externe_RapidJSON_doc_tutorial.zh-cn_TemporaryValues}
有时候,我们想直接构造一个\+Value并传递给一个“转移”函数(如{\ttfamily Push\+Back()}、{\ttfamily Add\+Member()})。由于临时对象是不能转换为正常的\+Value引用,我们加入了一个方便的{\ttfamily Move()}函数:


\begin{DoxyCode}
\hyperlink{class_generic_value}{Value} a(\hyperlink{rapidjson_8h_a1d1cfd8ffb84e947f82999c682b666a7af41527d6925efa3c5c3dadb23dfef7c8}{kArrayType});
\hyperlink{class_generic_document_a35155b912da66ced38d22e2551364c57}{Document::AllocatorType}& allocator = document.
      \hyperlink{class_generic_document_aa4609d6b19f86aec1a6b96edf2c27686}{GetAllocator}();
\textcolor{comment}{// a.PushBack(Value(42), allocator);       // 不能通过编译}
a.PushBack(\hyperlink{document_8h_a071cf97155ba72ac9a1fc4ad7e63d481}{Value}().SetInt(42), allocator); \textcolor{comment}{// fluent API}
a.PushBack(\hyperlink{document_8h_a071cf97155ba72ac9a1fc4ad7e63d481}{Value}(42).Move(), allocator);   \textcolor{comment}{// 和上一行相同}
\end{DoxyCode}
\hypertarget{md_Commun_Externe_RapidJSON_doc_tutorial.zh-cn_CreateString}{}\subsection{Create String}\label{md_Commun_Externe_RapidJSON_doc_tutorial.zh-cn_CreateString}
Rapid\+J\+S\+O\+N提供两个\+String的存储策略。


\begin{DoxyEnumerate}
\item copy-\/string\+: 分配缓冲区,然后把来源数据复制至它。
\item const-\/string\+: 简单地储存字符串的指针。
\end{DoxyEnumerate}

Copy-\/string总是安全的,因为它拥有数据的克隆。\+Const-\/string可用于存储字符串字面量,以及用于在\+D\+O\+M一节中将会提到的in-\/situ解析中。

为了让用户自定义内存分配方式,当一个操作可能需要内存分配时,\+Rapid\+J\+S\+O\+N要求用户传递一个allocator实例作为\+A\+P\+I参数。此设计避免了在每个\+Value存储allocator(或document)的指针。

因此,当我们把一个copy-\/string赋值时, 调用含有allocator的{\ttfamily Set\+String()}重载函数:


\begin{DoxyCode}
\hyperlink{class_generic_document}{Document} document;
\hyperlink{class_generic_value}{Value} author;
\textcolor{keywordtype}{char} buffer[10];
\textcolor{keywordtype}{int} len = sprintf(buffer, \textcolor{stringliteral}{"%s %s"}, \textcolor{stringliteral}{"Milo"}, \textcolor{stringliteral}{"Yip"}); \textcolor{comment}{// 动态创建的字符串。}
author.SetString(buffer, len, document.\hyperlink{class_generic_document_aa4609d6b19f86aec1a6b96edf2c27686}{GetAllocator}());
memset(buffer, 0, \textcolor{keyword}{sizeof}(buffer));
\textcolor{comment}{// 清空buffer后author.GetString() 仍然包含 "Milo Yip"}
\end{DoxyCode}


在此例子中,我们使用{\ttfamily Document}实例的allocator。这是使用\+Rapid\+J\+S\+O\+N时常用的惯用法。但你也可以用其他allocator实例。

另外,上面的{\ttfamily Set\+String()}需要长度参数。这个\+A\+P\+I能处理含有空字符的字符串。另一个{\ttfamily Set\+String()}重载函数没有长度参数,它假设输入是空字符结尾的,并会调用类似{\ttfamily strlen()}的函数去获取长度。

最后,对于字符串字面量或有安全生命周期的字符串,可以使用const-\/string版本的{\ttfamily Set\+String()},它没有allocator参数。对于字符串家面量(或字符数组常量),只需简单地传递字面量,又安全又高效:


\begin{DoxyCode}
\hyperlink{class_generic_value}{Value} s;
s.SetString(\textcolor{stringliteral}{"rapidjson"});    \textcolor{comment}{// 可包含空字符,长度在编译萁推导}
s = \textcolor{stringliteral}{"rapidjson"};             \textcolor{comment}{// 上行的缩写}
\end{DoxyCode}


对于字符指针,\+Rapid\+J\+S\+O\+N需要作一个标记,代表它不复制也是安全的。可以使用{\ttfamily String\+Ref}函数:


\begin{DoxyCode}
\textcolor{keyword}{const} \textcolor{keywordtype}{char} * cstr = getenv(\textcolor{stringliteral}{"USER"});
\textcolor{keywordtype}{size\_t} cstr\_len = ...;                 \textcolor{comment}{// 如果有长度}
\hyperlink{class_generic_value}{Value} s;
\textcolor{comment}{// s.SetString(cstr);                  // 这不能通过编译}
s.SetString(\hyperlink{struct_generic_string_ref_aa6b9fd9f6aa49405a574c362ba9af6b5}{StringRef}(cstr));          \textcolor{comment}{// 可以,假设它的生命周期案全,并且是以空字符结尾的}
s = \hyperlink{struct_generic_string_ref_aa6b9fd9f6aa49405a574c362ba9af6b5}{StringRef}(cstr);                   \textcolor{comment}{// 上行的缩写}
s.SetString(\hyperlink{struct_generic_string_ref_aa6b9fd9f6aa49405a574c362ba9af6b5}{StringRef}(cstr, cstr\_len));\textcolor{comment}{// 更快,可处理空字符}
s = \hyperlink{struct_generic_string_ref_aa6b9fd9f6aa49405a574c362ba9af6b5}{StringRef}(cstr, cstr\_len);         \textcolor{comment}{// 上行的缩写}
\end{DoxyCode}
\hypertarget{md_Commun_Externe_RapidJSON_doc_tutorial.zh-cn_ModifyArray}{}\subsection{Modify Array}\label{md_Commun_Externe_RapidJSON_doc_tutorial.zh-cn_ModifyArray}
Array类型的\+Value提供与{\ttfamily std\+::vector}相似的\+A\+P\+I。


\begin{DoxyItemize}
\item {\ttfamily Clear()}
\item {\ttfamily Reserve(\+Size\+Type, Allocator\&)}
\item {\ttfamily Value\& Push\+Back(\+Value\&, Allocator\&)}
\item {\ttfamily template $<$typename T$>$ \hyperlink{class_generic_value}{Generic\+Value}\& Push\+Back(\+T, Allocator\&)}
\item {\ttfamily Value\& Pop\+Back()}
\item {\ttfamily Value\+Iterator Erase(\+Const\+Value\+Iterator pos)}
\item {\ttfamily Value\+Iterator Erase(\+Const\+Value\+Iterator first, Const\+Value\+Iterator last)}
\end{DoxyItemize}

注意,{\ttfamily Reserve(...)}及{\ttfamily Push\+Back(...)}可能会为数组元素分配内存,所以需要一个allocator。

以下是{\ttfamily Push\+Back()}的例子:


\begin{DoxyCode}
\hyperlink{class_generic_value}{Value} a(\hyperlink{rapidjson_8h_a1d1cfd8ffb84e947f82999c682b666a7af41527d6925efa3c5c3dadb23dfef7c8}{kArrayType});
\hyperlink{class_generic_document_a35155b912da66ced38d22e2551364c57}{Document::AllocatorType}& allocator = document.
      \hyperlink{class_generic_document_aa4609d6b19f86aec1a6b96edf2c27686}{GetAllocator}();

\textcolor{keywordflow}{for} (\textcolor{keywordtype}{int} i = 5; i <= 10; i++)
    a.PushBack(i, allocator);   \textcolor{comment}{// 可能需要调用realloc()所以需要allocator}

\textcolor{comment}{// 流畅接口(Fluent interface)}
a.PushBack(\textcolor{stringliteral}{"Lua"}, allocator).PushBack(\textcolor{stringliteral}{"Mio"}, allocator);
\end{DoxyCode}


与\+S\+T\+L不一样的是,{\ttfamily Push\+Back()}/{\ttfamily Pop\+Back()}返回\+Array本身的引用。这称为流畅接口(\+\_\+fluent interface\+\_\+)。

如果你想在\+Array中加入一个非常量字符串,或是一个没有足够生命周期的字符串(见\href{#CreateString}{\tt Create String}),你需要使用copy-\/string A\+P\+I去创建一个\+String。为了避免加入中间变量,可以就地使用一个\href{#TemporaryValues}{\tt 临时值}:


\begin{DoxyCode}
\textcolor{comment}{// 就地Value参数}
contact.PushBack(\hyperlink{document_8h_a071cf97155ba72ac9a1fc4ad7e63d481}{Value}(\textcolor{stringliteral}{"copy"}, document.\hyperlink{class_generic_document_aa4609d6b19f86aec1a6b96edf2c27686}{GetAllocator}()).Move(), \textcolor{comment}{// copy string}
                 document.\hyperlink{class_generic_document_aa4609d6b19f86aec1a6b96edf2c27686}{GetAllocator}());

\textcolor{comment}{// 显式Value参数}
\hyperlink{class_generic_value}{Value} val(\textcolor{stringliteral}{"key"}, document.\hyperlink{class_generic_document_aa4609d6b19f86aec1a6b96edf2c27686}{GetAllocator}()); \textcolor{comment}{// copy string}
contact.PushBack(val, document.\hyperlink{class_generic_document_aa4609d6b19f86aec1a6b96edf2c27686}{GetAllocator}());
\end{DoxyCode}
\hypertarget{md_Commun_Externe_RapidJSON_doc_tutorial.zh-cn_ModifyObject}{}\subsection{Modify Object}\label{md_Commun_Externe_RapidJSON_doc_tutorial.zh-cn_ModifyObject}
Object是键值对的集合。每个键必须为\+String。要修改\+Object,方法是增加或移除成员。以下的\+A\+P\+I用来增加城员:


\begin{DoxyItemize}
\item {\ttfamily Value\& Add\+Member(\+Value\&, Value\&, Allocator\& allocator)}
\item {\ttfamily Value\& Add\+Member(\+String\+Ref\+Type, Value\&, Allocator\&)}
\item {\ttfamily template $<$typename T$>$ Value\& Add\+Member(\+String\+Ref\+Type, T value, Allocator\&)}
\end{DoxyItemize}

以下是一个例子。


\begin{DoxyCode}
\hyperlink{class_generic_value}{Value} contact(kObject);
contact.AddMember(\textcolor{stringliteral}{"name"}, \textcolor{stringliteral}{"Milo"}, document.\hyperlink{class_generic_document_aa4609d6b19f86aec1a6b96edf2c27686}{GetAllocator}());
contact.AddMember(\textcolor{stringliteral}{"married"}, \textcolor{keyword}{true}, document.\hyperlink{class_generic_document_aa4609d6b19f86aec1a6b96edf2c27686}{GetAllocator}());
\end{DoxyCode}


使用{\ttfamily String\+Ref\+Type}作为name参数的重载版本与字符串的{\ttfamily Set\+String}的接口相似。 这些重载是为了避免复制{\ttfamily name}字符串,因为\+J\+S\+ON object中经常会使用常数键名。

如果你需要从非常数字符串或生命周期不足的字符串创建键名(见\href{#CreateString}{\tt 创建\+String}),你需要使用copy-\/string A\+P\+I。为了避免中间变量,可以就地使用\href{#TemporaryValues}{\tt 临时值}:


\begin{DoxyCode}
\textcolor{comment}{// 就地Value参数}
contact.AddMember(\hyperlink{document_8h_a071cf97155ba72ac9a1fc4ad7e63d481}{Value}(\textcolor{stringliteral}{"copy"}, document.\hyperlink{class_generic_document_aa4609d6b19f86aec1a6b96edf2c27686}{GetAllocator}()).Move(), \textcolor{comment}{// copy string}
                  \hyperlink{document_8h_a071cf97155ba72ac9a1fc4ad7e63d481}{Value}().Move(),                                \textcolor{comment}{// null value}
                  document.\hyperlink{class_generic_document_aa4609d6b19f86aec1a6b96edf2c27686}{GetAllocator}());

\textcolor{comment}{// 显式参数}
\hyperlink{class_generic_value}{Value} key(\textcolor{stringliteral}{"key"}, document.\hyperlink{class_generic_document_aa4609d6b19f86aec1a6b96edf2c27686}{GetAllocator}()); \textcolor{comment}{// copy string name}
\hyperlink{class_generic_value}{Value} val(42);                             \textcolor{comment}{// 某Value}
contact.AddMember(key, val, document.\hyperlink{class_generic_document_aa4609d6b19f86aec1a6b96edf2c27686}{GetAllocator}());
\end{DoxyCode}


移除成员有几个选择:


\begin{DoxyItemize}
\item {\ttfamily bool Remove\+Member(const Ch$\ast$ name)}:使用键名来移除成员(线性时间复杂度)。
\item {\ttfamily bool Remove\+Member(const Value\& name)}:除了{\ttfamily name}是一个\+Value,和上一行相同。
\item {\ttfamily Member\+Iterator Remove\+Member(\+Member\+Iterator)}:使用迭待器移除成员(\+\_\+常数\+\_\+时间复杂度)。
\item {\ttfamily Member\+Iterator Erase\+Member(\+Member\+Iterator)}:和上行相似但维持成员次序(线性时间复杂度)。
\item {\ttfamily Member\+Iterator Erase\+Member(\+Member\+Iterator first, Member\+Iterator last)}:移除一个范围内的成员,维持次序(线性时间复杂度)。
\end{DoxyItemize}

{\ttfamily Member\+Iterator Remove\+Member(\+Member\+Iterator)}使用了“转移最后”手法来达成常数时间复杂度。基本上就是析构迭代器位置的成员,然后把最后的成员转移至迭代器位置。因此,成员的次序会被改变。\hypertarget{md_Commun_Externe_RapidJSON_doc_tutorial.zh-cn_DeepCopyValue}{}\subsection{Deep Copy Value}\label{md_Commun_Externe_RapidJSON_doc_tutorial.zh-cn_DeepCopyValue}
若我们真的要复制一个\+D\+O\+M树,我们可使用两个\+A\+P\+Is作深复制:含allocator的构造函数及{\ttfamily Copy\+From()}。


\begin{DoxyCode}
\hyperlink{class_generic_document}{Document} d;
\hyperlink{class_generic_document_a35155b912da66ced38d22e2551364c57}{Document::AllocatorType}& a = d.\hyperlink{class_generic_document_aa4609d6b19f86aec1a6b96edf2c27686}{GetAllocator}();
\hyperlink{class_generic_value}{Value} v1(\textcolor{stringliteral}{"foo"});
\textcolor{comment}{// Value v2(v1); // 不容许}

\hyperlink{class_generic_value}{Value} v2(v1, a);                      \textcolor{comment}{// 制造一个克隆}
assert(v1.IsString());                \textcolor{comment}{// v1不变}
d.SetArray().PushBack(v1, a).PushBack(v2, a);
assert(v1.IsNull() && v2.IsNull());   \textcolor{comment}{// 两个都转移动d}

v2.CopyFrom(d, a);                    \textcolor{comment}{// 把整个document复制至v2}
assert(d.IsArray() && d.Size() == 2); \textcolor{comment}{// d不变}
v1.SetObject().AddMember(\textcolor{stringliteral}{"array"}, v2, a);
d.PushBack(v1, a);
\end{DoxyCode}
\hypertarget{md_Commun_Externe_RapidJSON_doc_tutorial.zh-cn_SwapValues}{}\subsection{Swap Values}\label{md_Commun_Externe_RapidJSON_doc_tutorial.zh-cn_SwapValues}
Rapid\+J\+S\+O\+N也提供{\ttfamily Swap()}。


\begin{DoxyCode}
\hyperlink{class_generic_value}{Value} a(123);
\hyperlink{class_generic_value}{Value} b(\textcolor{stringliteral}{"Hello"});
a.Swap(b);
assert(a.IsString());
assert(b.IsInt());
\end{DoxyCode}


无论两棵\+D\+O\+M树有多复杂,交换是很快的(常数时间)。\hypertarget{md_Commun_Externe_RapidJSON_doc_tutorial.zh-cn_WhatsNext}{}\section{What\textquotesingle{}s next}\label{md_Commun_Externe_RapidJSON_doc_tutorial.zh-cn_WhatsNext}
本教程展示了如何询查及修改\+D\+O\+M树。\+Rapid\+J\+S\+O\+N还有一个重要概念:


\begin{DoxyEnumerate}
\item 流 是读写\+J\+S\+O\+N的通道。流可以是内存字符串、文件流等。用户也可以自定义流。
\item 编码定义在流或内存中使用的字符编码。\+Rapid\+J\+S\+O\+N也在内部提供\+Unicode转换及校验功能。
\item D\+OM的基本功能已在本教程里介绍。还有更高级的功能,如原位($\ast$in situ$\ast$)解析、其他解析选项及高级用法。
\item S\+AX 是\+Rapid\+J\+S\+O\+N解析/生成功能的基础。学习使用{\ttfamily Reader}/{\ttfamily \hyperlink{class_writer}{Writer}}去实现更高性能的应用程序。也可以使用{\ttfamily \hyperlink{class_pretty_writer}{Pretty\+Writer}}去格式化\+J\+S\+O\+N。
\item 性能展示一些我们做的及第三方的性能测试。
\item 技术内幕讲述一些\+Rapid\+J\+S\+O\+N内部的设计及技术。
\end{DoxyEnumerate}

你也可以参考常见问题、\+A\+P\+I文档、例子及单元测试。 