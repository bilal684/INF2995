According to \href{http://www.ecma-international.org/publications/files/ECMA-ST/ECMA-404.pdf}{\tt E\+C\+M\+A-\/404},

\begin{quote}
(in Introduction) J\+S\+ON text is a sequence of Unicode code points. \end{quote}


The earlier \href{http://www.ietf.org/rfc/rfc4627.txt}{\tt R\+F\+C4627} stated that,

\begin{quote}
(in §3) J\+S\+ON text S\+H\+A\+LL be encoded in Unicode. The default encoding is U\+T\+F-\/8. \end{quote}


\begin{quote}
(in §6) J\+S\+ON may be represented using U\+T\+F-\/8, U\+T\+F-\/16, or U\+T\+F-\/32. When J\+S\+ON is written in U\+T\+F-\/8, J\+S\+ON is 8bit compatible. When J\+S\+ON is written in U\+T\+F-\/16 or U\+T\+F-\/32, the binary content-\/transfer-\/encoding must be used. \end{quote}


Rapid\+J\+S\+ON supports various encodings. It can also validate the encodings of J\+S\+ON, and transconding J\+S\+ON among encodings. All these features are implemented internally, without the need for external libraries (e.\+g. \href{http://site.icu-project.org/}{\tt I\+CU}).\hypertarget{md_Cadriciel_Commun_Externe_RapidJSON_doc_encoding.zh-cn_Unicode}{}\section{Unicode}\label{md_Cadriciel_Commun_Externe_RapidJSON_doc_encoding.zh-cn_Unicode}
From \href{http://www.unicode.org/standard/WhatIsUnicode.html}{\tt Unicode\textquotesingle{}s official website}\+: \begin{quote}
Unicode provides a unique number for every character, no matter what the platform, no matter what the program, no matter what the language. \end{quote}


Those unique numbers are called code points, which is in the range {\ttfamily 0x0} to {\ttfamily 0x10\+F\+F\+FF}.\hypertarget{md_Cadriciel_Commun_Externe_RapidJSON_doc_encoding.zh-cn_UTF}{}\subsection{Unicode Transformation Format}\label{md_Cadriciel_Commun_Externe_RapidJSON_doc_encoding.zh-cn_UTF}
There are various encodings for storing Unicode code points. These are called Unicode Transformation Format (U\+TF). Rapid\+J\+S\+ON supports the most commonly used U\+T\+Fs, including


\begin{DoxyItemize}
\item U\+T\+F-\/8\+: 8-\/bit variable-\/width encoding. It maps a code point to 1–4 bytes.
\item U\+T\+F-\/16\+: 16-\/bit variable-\/width encoding. It maps a code point to 1–2 16-\/bit code units (i.\+e., 2–4 bytes).
\item U\+T\+F-\/32\+: 32-\/bit fixed-\/width encoding. It directly maps a code point to a single 32-\/bit code unit (i.\+e. 4 bytes).
\end{DoxyItemize}

For U\+T\+F-\/16 and U\+T\+F-\/32, the byte order (endianness) does matter. Within computer memory, they are often stored in the computer\textquotesingle{}s endianness. However, when it is stored in file or transferred over network, we need to state the byte order of the byte sequence, either little-\/endian (LE) or big-\/endian (BE).

Rapid\+J\+S\+ON provide these encodings via the structs in {\ttfamily \hyperlink{encodings_8h_source}{rapidjson/encodings.\+h}}\+:


\begin{DoxyCode}
\textcolor{keyword}{namespace }\hyperlink{namespacerapidjson}{rapidjson} \{

\textcolor{keyword}{template}<\textcolor{keyword}{typename} CharType = \textcolor{keywordtype}{char}>
\textcolor{keyword}{struct }\hyperlink{struct_u_t_f8}{UTF8};

\textcolor{keyword}{template}<\textcolor{keyword}{typename} CharType = \textcolor{keywordtype}{wchar\_t}>
\textcolor{keyword}{struct }\hyperlink{struct_u_t_f16}{UTF16};

\textcolor{keyword}{template}<\textcolor{keyword}{typename} CharType = \textcolor{keywordtype}{wchar\_t}>
\textcolor{keyword}{struct }\hyperlink{struct_u_t_f16_l_e}{UTF16LE};

\textcolor{keyword}{template}<\textcolor{keyword}{typename} CharType = \textcolor{keywordtype}{wchar\_t}>
\textcolor{keyword}{struct }\hyperlink{struct_u_t_f16_b_e}{UTF16BE};

\textcolor{keyword}{template}<\textcolor{keyword}{typename} CharType = \textcolor{keywordtype}{unsigned}>
\textcolor{keyword}{struct }\hyperlink{struct_u_t_f32}{UTF32};

\textcolor{keyword}{template}<\textcolor{keyword}{typename} CharType = \textcolor{keywordtype}{unsigned}>
\textcolor{keyword}{struct }\hyperlink{struct_u_t_f32_l_e}{UTF32LE};

\textcolor{keyword}{template}<\textcolor{keyword}{typename} CharType = \textcolor{keywordtype}{unsigned}>
\textcolor{keyword}{struct }\hyperlink{struct_u_t_f32_b_e}{UTF32BE};

\} \textcolor{comment}{// namespace rapidjson}
\end{DoxyCode}


For processing text in memory, we normally use {\ttfamily \hyperlink{struct_u_t_f8}{U\+T\+F8}}, {\ttfamily \hyperlink{struct_u_t_f16}{U\+T\+F16}} or {\ttfamily \hyperlink{struct_u_t_f32}{U\+T\+F32}}. For processing text via I/O, we may use {\ttfamily \hyperlink{struct_u_t_f8}{U\+T\+F8}}, {\ttfamily \hyperlink{struct_u_t_f16_l_e}{U\+T\+F16\+LE}}, {\ttfamily \hyperlink{struct_u_t_f16_b_e}{U\+T\+F16\+BE}}, {\ttfamily \hyperlink{struct_u_t_f32_l_e}{U\+T\+F32\+LE}} or {\ttfamily \hyperlink{struct_u_t_f32_b_e}{U\+T\+F32\+BE}}.

When using the D\+O\+M-\/style A\+PI, the {\ttfamily Encoding} template parameter in {\ttfamily \hyperlink{class_generic_value}{Generic\+Value}$<$Encoding$>$} and {\ttfamily \hyperlink{class_generic_document}{Generic\+Document}$<$Encoding$>$} indicates the encoding to be used to represent J\+S\+ON string in memory. So normally we will use {\ttfamily \hyperlink{struct_u_t_f8}{U\+T\+F8}}, {\ttfamily \hyperlink{struct_u_t_f16}{U\+T\+F16}} or {\ttfamily \hyperlink{struct_u_t_f32}{U\+T\+F32}} for this template parameter. The choice depends on operating systems and other libraries that the application is using. For example, Windows A\+PI represents Unicode characters in U\+T\+F-\/16, while most Linux distributions and applications prefer U\+T\+F-\/8.

Example of U\+T\+F-\/16 D\+OM declaration\+:


\begin{DoxyCode}
\textcolor{keyword}{typedef} \hyperlink{class_generic_document}{GenericDocument<UTF16<>} > WDocument;
\textcolor{keyword}{typedef} \hyperlink{class_generic_value}{GenericValue<UTF16<>} > WValue;
\end{DoxyCode}


For a detail example, please check the example in D\+OM\textquotesingle{}s Encoding section.\hypertarget{md_Cadriciel_Commun_Externe_RapidJSON_doc_encoding.zh-cn_CharacterType}{}\subsection{Character Type}\label{md_Cadriciel_Commun_Externe_RapidJSON_doc_encoding.zh-cn_CharacterType}
As shown in the declaration, each encoding has a {\ttfamily Char\+Type} template parameter. Actually, it may be a little bit confusing, but each {\ttfamily Char\+Type} stores a code unit, not a character (code point). As mentioned in previous section, a code point may be encoded to 1–4 code units for U\+T\+F-\/8.

For {\ttfamily \hyperlink{struct_u_t_f16}{U\+T\+F16}(L\+E$\vert$\+BE)}, {\ttfamily \hyperlink{struct_u_t_f32}{U\+T\+F32}(L\+E$\vert$\+BE)}, the {\ttfamily Char\+Type} must be integer type of at least 2 and 4 bytes respectively.

Note that C++11 introduces {\ttfamily char16\+\_\+t} and {\ttfamily char32\+\_\+t}, which can be used for {\ttfamily \hyperlink{struct_u_t_f16}{U\+T\+F16}} and {\ttfamily \hyperlink{struct_u_t_f32}{U\+T\+F32}} respectively.\hypertarget{md_Cadriciel_Commun_Externe_RapidJSON_doc_encoding.zh-cn_AutoUTF}{}\subsection{Auto\+U\+TF}\label{md_Cadriciel_Commun_Externe_RapidJSON_doc_encoding.zh-cn_AutoUTF}
Previous encodings are statically bound in compile-\/time. In other words, user must know exactly which encodings will be used in the memory or streams. However, sometimes we may need to read/write files of different encodings. The encoding needed to be decided in runtime.

{\ttfamily \hyperlink{struct_auto_u_t_f}{Auto\+U\+TF}} is an encoding designed for this purpose. It chooses which encoding to be used according to the input or output stream. Currently, it should be used with {\ttfamily \hyperlink{class_encoded_input_stream}{Encoded\+Input\+Stream}} and {\ttfamily \hyperlink{class_encoded_output_stream}{Encoded\+Output\+Stream}}.\hypertarget{md_Cadriciel_Commun_Externe_RapidJSON_doc_encoding.zh-cn_ASCII}{}\subsection{A\+S\+C\+II}\label{md_Cadriciel_Commun_Externe_RapidJSON_doc_encoding.zh-cn_ASCII}
Although the J\+S\+ON standards did not mention about \href{http://en.wikipedia.org/wiki/ASCII}{\tt A\+S\+C\+II}, sometimes we would like to write 7-\/bit \hyperlink{struct_a_s_c_i_i}{A\+S\+C\+II} J\+S\+ON for applications that cannot handle U\+T\+F-\/8. Since any J\+S\+ON can represent unicode characters in escaped sequence {\ttfamily \textbackslash{}u\+X\+X\+XX}, J\+S\+ON can always be encoded in \hyperlink{struct_a_s_c_i_i}{A\+S\+C\+II}.

Here is an example for writing a U\+T\+F-\/8 D\+OM into \hyperlink{struct_a_s_c_i_i}{A\+S\+C\+II}\+:


\begin{DoxyCode}
\textcolor{keyword}{using namespace }\hyperlink{namespacerapidjson}{rapidjson};
\hyperlink{class_generic_document}{Document} d; \textcolor{comment}{// UTF8<>}
\textcolor{comment}{// ...}
\hyperlink{class_generic_string_buffer}{StringBuffer} buffer;
\hyperlink{class_writer}{Writer<StringBuffer, Document::EncodingType, ASCII<>} > 
      writer(buffer);
d.Accept(writer);
std::cout << buffer.GetString();
\end{DoxyCode}


\hyperlink{struct_a_s_c_i_i}{A\+S\+C\+II} can be used in input stream. If the input stream contains bytes with values above 127, it will cause {\ttfamily k\+Parse\+Error\+String\+Invalid\+Encoding} error.

\hyperlink{struct_a_s_c_i_i}{A\+S\+C\+II} {\itshape cannot} be used in memory (encoding of {\ttfamily Document} or target encoding of {\ttfamily Reader}), as it cannot represent Unicode code points.\hypertarget{md_Cadriciel_Commun_Externe_RapidJSON_doc_encoding.zh-cn_ValidationTranscoding}{}\section{Validation \& Transcoding}\label{md_Cadriciel_Commun_Externe_RapidJSON_doc_encoding.zh-cn_ValidationTranscoding}
When Rapid\+J\+S\+ON parses a J\+S\+ON, it can validate the input J\+S\+ON, whether it is a valid sequence of a specified encoding. This option can be turned on by adding {\ttfamily k\+Parse\+Validate\+Encoding\+Flag} in {\ttfamily parse\+Flags} template parameter.

If the input encoding and output encoding is different, {\ttfamily Reader} and {\ttfamily \hyperlink{class_writer}{Writer}} will automatically transcode (convert) the text. In this case, {\ttfamily k\+Parse\+Validate\+Encoding\+Flag} is not necessary, as it must decode the input sequence. And if the sequence was unable to be decoded, it must be invalid.\hypertarget{md_Cadriciel_Commun_Externe_RapidJSON_doc_encoding.zh-cn_Transcoder}{}\subsection{Transcoder}\label{md_Cadriciel_Commun_Externe_RapidJSON_doc_encoding.zh-cn_Transcoder}
Although the encoding functions in Rapid\+J\+S\+ON are designed for J\+S\+ON parsing/generation, user may abuse them for transcoding of non-\/\+J\+S\+ON strings.

Here is an example for transcoding a string from U\+T\+F-\/8 to U\+T\+F-\/16\+:


\begin{DoxyCode}
\textcolor{preprocessor}{#include "rapidjson/encodings.h"}

\textcolor{keyword}{using namespace }\hyperlink{namespacerapidjson}{rapidjson};

\textcolor{keyword}{const} \textcolor{keywordtype}{char}* s = \textcolor{stringliteral}{"..."}; \textcolor{comment}{// UTF-8 string}
\hyperlink{struct_generic_string_stream}{StringStream} source(s);
\hyperlink{class_generic_string_buffer}{GenericStringBuffer<UTF16<>} > target;

\textcolor{keywordtype}{bool} hasError = \textcolor{keyword}{false};
\textcolor{keywordflow}{while} (source.Peak() != \textcolor{charliteral}{'\(\backslash\)0'})
    \textcolor{keywordflow}{if} (!\hyperlink{struct_transcoder_a0ea2edfe35784ebf1063921d2bd5fb66}{Transcoder::Transcode}<\hyperlink{struct_u_t_f8}{UTF8<>}, \hyperlink{struct_u_t_f16}{UTF16<>} >(source, target)) \{
        hasError = \textcolor{keyword}{true};
        \textcolor{keywordflow}{break};
    \}

\textcolor{keywordflow}{if} (!hasError) \{
    \textcolor{keyword}{const} \textcolor{keywordtype}{wchar\_t}* t = target.GetString();
    \textcolor{comment}{// ...}
\}
\end{DoxyCode}


You may also use {\ttfamily \hyperlink{struct_auto_u_t_f}{Auto\+U\+TF}} and the associated streams for setting source/target encoding in runtime. 