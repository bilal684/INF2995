在\+Rapid\+J\+S\+O\+N中,{\ttfamily \hyperlink{classrapidjson_1_1_stream}{rapidjson\+::\+Stream}}是用於读写\+J\+S\+O\+N的概念(概念是指\+C++的concept)。在这里我们先介绍如何使用\+Rapid\+J\+S\+O\+N提供的各种流。然后再看看如何自行定义流。\hypertarget{md_Cadriciel_Commun_Externe_RapidJSON_doc_stream.zh-cn_MemoryStreams}{}\section{Memory Streams}\label{md_Cadriciel_Commun_Externe_RapidJSON_doc_stream.zh-cn_MemoryStreams}
内存流把\+J\+S\+O\+N存储在内存之中。\hypertarget{md_Cadriciel_Commun_Externe_RapidJSON_doc_stream.zh-cn_StringStream}{}\subsection{String\+Stream (\+Input)}\label{md_Cadriciel_Commun_Externe_RapidJSON_doc_stream.zh-cn_StringStream}
{\ttfamily String\+Stream}是最基本的输入流,它表示一个完整的、只读的、存储于内存的\+J\+S\+O\+N。它在{\ttfamily \hyperlink{rapidjson_8h}{rapidjson/rapidjson.\+h}}中定义。


\begin{DoxyCode}
\textcolor{preprocessor}{#include "\hyperlink{document_8h}{rapidjson/document.h}"} \textcolor{comment}{// 会包含 "rapidjson/rapidjson.h"}

\textcolor{keyword}{using namespace }\hyperlink{namespacerapidjson}{rapidjson};

\textcolor{comment}{// ...}
\textcolor{keyword}{const} \textcolor{keywordtype}{char} json[] = \textcolor{stringliteral}{"[1, 2, 3, 4]"};
\hyperlink{struct_generic_string_stream}{StringStream} s(json);

\hyperlink{class_generic_document}{Document} d;
d.\hyperlink{class_generic_document_afe94c0abc83a20f2d7dc1ba7677e6238}{ParseStream}(s);
\end{DoxyCode}


由于这是非常常用的用法,\+Rapid\+J\+S\+O\+N提供{\ttfamily Document\+::\+Parse(const char$\ast$)}去做完全相同的事情:


\begin{DoxyCode}
\textcolor{comment}{// ...}
\textcolor{keyword}{const} \textcolor{keywordtype}{char} json[] = \textcolor{stringliteral}{"[1, 2, 3, 4]"};
\hyperlink{class_generic_document}{Document} d;
d.\hyperlink{class_generic_document_aebd4e7fddd80c1e1174837aee6d2159b}{Parse}(json);
\end{DoxyCode}


需要注意,{\ttfamily String\+Stream}是{\ttfamily \hyperlink{struct_generic_string_stream}{Generic\+String\+Stream}$<$\hyperlink{struct_u_t_f8}{U\+T\+F8}$<$$>$ $>$}的typedef,使用者可用其他编码类去代表流所使用的字符集。\hypertarget{md_Cadriciel_Commun_Externe_RapidJSON_doc_stream.zh-cn_StringBuffer}{}\subsection{String\+Buffer (\+Output)}\label{md_Cadriciel_Commun_Externe_RapidJSON_doc_stream.zh-cn_StringBuffer}
{\ttfamily String\+Buffer}是一个简单的输出流。它分配一个内存缓冲区,供写入整个\+J\+S\+O\+N。可使用{\ttfamily Get\+String()}来获取该缓冲区。


\begin{DoxyCode}
\textcolor{preprocessor}{#include "rapidjson/stringbuffer.h"}

\hyperlink{class_generic_string_buffer}{StringBuffer} buffer;
\hyperlink{class_writer}{Writer<StringBuffer>} writer(buffer);
d.Accept(writer);

\textcolor{keyword}{const} \textcolor{keywordtype}{char}* output = buffer.GetString();
\end{DoxyCode}


当缓冲区满溢,它将自动增加容量。缺省容量是256个字符(\+U\+T\+F8是256字节,\+U\+T\+F16是512字节等)。使用者能自行提供分配器及初始容量。


\begin{DoxyCode}
\hyperlink{class_generic_string_buffer}{StringBuffer} buffer1(0, 1024); \textcolor{comment}{// 使用它的分配器,初始大小 = 1024}
\hyperlink{class_generic_string_buffer}{StringBuffer} buffer2(allocator, 1024);
\end{DoxyCode}


如无设置分配器,{\ttfamily String\+Buffer}会自行实例化一个内部分配器。

相似地,{\ttfamily String\+Buffer}是{\ttfamily \hyperlink{class_generic_string_buffer}{Generic\+String\+Buffer}$<$\hyperlink{struct_u_t_f8}{U\+T\+F8}$<$$>$ $>$}的typedef。\hypertarget{md_Cadriciel_Commun_Externe_RapidJSON_doc_stream.zh-cn_FileStreams}{}\section{File Streams}\label{md_Cadriciel_Commun_Externe_RapidJSON_doc_stream.zh-cn_FileStreams}
当要从文件解析一个\+J\+S\+O\+N,你可以把整个\+J\+S\+O\+N读入内存并使用上述的{\ttfamily String\+Stream}。

然而,若\+J\+S\+O\+N很大,或是内存有限,你可以改用{\ttfamily \hyperlink{class_file_read_stream}{File\+Read\+Stream}}。它只会从文件读取一部分至缓冲区,然后让那部分被解析。若缓冲区的字符都被读完,它会再从文件读取下一部分。\hypertarget{md_Cadriciel_Commun_Externe_RapidJSON_doc_stream.zh-cn_FileReadStream}{}\subsection{File\+Read\+Stream (\+Input)}\label{md_Cadriciel_Commun_Externe_RapidJSON_doc_stream.zh-cn_FileReadStream}
{\ttfamily \hyperlink{class_file_read_stream}{File\+Read\+Stream}}通过{\ttfamily F\+I\+LE}指针读取文件。使用者需要提供一个缓冲区。


\begin{DoxyCode}
\textcolor{preprocessor}{#include "rapidjson/filereadstream.h"}
\textcolor{preprocessor}{#include <cstdio>}

\textcolor{keyword}{using namespace }\hyperlink{namespacerapidjson}{rapidjson};

FILE* fp = fopen(\textcolor{stringliteral}{"big.json"}, \textcolor{stringliteral}{"rb"}); \textcolor{comment}{// 非Windows平台使用"r"}

\textcolor{keywordtype}{char} readBuffer[65536];
\hyperlink{class_file_read_stream}{FileReadStream} is(fp, readBuffer, \textcolor{keyword}{sizeof}(readBuffer));

\hyperlink{class_generic_document}{Document} d;
d.\hyperlink{class_generic_document_afe94c0abc83a20f2d7dc1ba7677e6238}{ParseStream}(is);

fclose(fp);
\end{DoxyCode}


与{\ttfamily String\+Streams}不一样,{\ttfamily \hyperlink{class_file_read_stream}{File\+Read\+Stream}}是一个字节流。它不处理编码。若文件并非\+U\+T\+F-\/8编码,可以把字节流用{\ttfamily \hyperlink{class_encoded_input_stream}{Encoded\+Input\+Stream}}包装。我们很快会讨论这个问题。

除了读取文件,使用者也可以使用{\ttfamily \hyperlink{class_file_read_stream}{File\+Read\+Stream}}来读取{\ttfamily stdin}。\hypertarget{md_Cadriciel_Commun_Externe_RapidJSON_doc_stream.zh-cn_FileWriteStream}{}\subsection{File\+Write\+Stream (\+Output)}\label{md_Cadriciel_Commun_Externe_RapidJSON_doc_stream.zh-cn_FileWriteStream}
{\ttfamily \hyperlink{class_file_write_stream}{File\+Write\+Stream}}是一个含缓冲功能的输出流。它的用法与{\ttfamily \hyperlink{class_file_read_stream}{File\+Read\+Stream}}非常相似。


\begin{DoxyCode}
\textcolor{preprocessor}{#include "rapidjson/filewritestream.h"}
\textcolor{preprocessor}{#include <cstdio>}

\textcolor{keyword}{using namespace }\hyperlink{namespacerapidjson}{rapidjson};

\hyperlink{class_generic_document}{Document} d;
d.\hyperlink{class_generic_document_aebd4e7fddd80c1e1174837aee6d2159b}{Parse}(json);
\textcolor{comment}{// ...}

FILE* fp = fopen(\textcolor{stringliteral}{"output.json"}, \textcolor{stringliteral}{"wb"}); \textcolor{comment}{// 非Windows平台使用"w"}

\textcolor{keywordtype}{char} writeBuffer[65536];
\hyperlink{class_file_write_stream}{FileWriteStream} os(fp, writeBuffer, \textcolor{keyword}{sizeof}(writeBuffer));

\hyperlink{class_writer}{Writer<FileWriteStream>} writer(os);
d.Accept(writer);

fclose(fp);
\end{DoxyCode}


它也可以把输出导向{\ttfamily stdout}。\hypertarget{md_Cadriciel_Commun_Externe_RapidJSON_doc_stream.zh-cn_EncodedStreams}{}\section{Encoded Streams}\label{md_Cadriciel_Commun_Externe_RapidJSON_doc_stream.zh-cn_EncodedStreams}
编码流(encoded streams)本身不存储\+J\+S\+O\+N,它们是通过包装字节流来提供基本的编码/解码功能。

如上所述,我们可以直接读入\+U\+T\+F-\/8字节流。然而,\+U\+T\+F-\/16及\+U\+T\+F-\/32有字节序(endian)问题。要正确地处理字节序,需要在读取时把字节转换成字符(如对\+U\+T\+F-\/16使用{\ttfamily wchar\+\_\+t}),以及在写入时把字符转换为字节。

除此以外,我们也需要处理\href{http://en.wikipedia.org/wiki/Byte_order_mark}{\tt 字节顺序标记(byte order mark, B\+O\+M)}。当从一个字节流读取时,需要检测\+B\+O\+M,或者仅仅是把存在的\+B\+O\+M消去。当把\+J\+S\+O\+N写入字节流时,也可选择写入\+B\+O\+M。

若一个流的编码在编译期已知,你可使用{\ttfamily \hyperlink{class_encoded_input_stream}{Encoded\+Input\+Stream}}及{\ttfamily \hyperlink{class_encoded_output_stream}{Encoded\+Output\+Stream}}。若一个流可能存储\+U\+T\+F-\/8、\+U\+T\+F-\/16\+L\+E、\+U\+T\+F-\/16\+B\+E、\+U\+T\+F-\/32\+L\+E、\+U\+T\+F-\/32\+B\+E的\+J\+S\+O\+N,并且编码只能在运行时得知,你便可以使用{\ttfamily \hyperlink{class_auto_u_t_f_input_stream}{Auto\+U\+T\+F\+Input\+Stream}}及{\ttfamily \hyperlink{class_auto_u_t_f_output_stream}{Auto\+U\+T\+F\+Output\+Stream}}。这些流定义在{\ttfamily \hyperlink{encodedstream_8h_source}{rapidjson/encodedstream.\+h}}。

注意到,这些编码流可以施于文件以外的流。例如,你可以用编码流包装内存中的文件或自定义的字节流。\hypertarget{md_Cadriciel_Commun_Externe_RapidJSON_doc_stream.zh-cn_EncodedInputStream}{}\subsection{Encoded\+Input\+Stream}\label{md_Cadriciel_Commun_Externe_RapidJSON_doc_stream.zh-cn_EncodedInputStream}
{\ttfamily \hyperlink{class_encoded_input_stream}{Encoded\+Input\+Stream}}含两个模板参数。第一个是{\ttfamily Encoding}类型,例如定义于{\ttfamily \hyperlink{encodings_8h_source}{rapidjson/encodings.\+h}}的{\ttfamily \hyperlink{struct_u_t_f8}{U\+T\+F8}}、{\ttfamily \hyperlink{struct_u_t_f16_l_e}{U\+T\+F16\+LE}}。第二个参数是被包装的流的类型。


\begin{DoxyCode}
\textcolor{preprocessor}{#include "\hyperlink{document_8h}{rapidjson/document.h}"}
\textcolor{preprocessor}{#include "rapidjson/filereadstream.h"}   \textcolor{comment}{// FileReadStream}
\textcolor{preprocessor}{#include "rapidjson/encodedstream.h"}    \textcolor{comment}{// EncodedInputStream}
\textcolor{preprocessor}{#include <cstdio>}

\textcolor{keyword}{using namespace }\hyperlink{namespacerapidjson}{rapidjson};

FILE* fp = fopen(\textcolor{stringliteral}{"utf16le.json"}, \textcolor{stringliteral}{"rb"}); \textcolor{comment}{// 非Windows平台使用"r"}

\textcolor{keywordtype}{char} readBuffer[256];
\hyperlink{class_file_read_stream}{FileReadStream} bis(fp, readBuffer, \textcolor{keyword}{sizeof}(readBuffer));

\hyperlink{class_encoded_input_stream}{EncodedInputStream<UTF16LE<>}, \hyperlink{class_file_read_stream}{FileReadStream}> eis(bis);  \textcolor{comment}{//
       用eis包装bis}

\hyperlink{class_generic_document}{Document} d; \textcolor{comment}{// Document为GenericDocument<UTF8<> > }
d.\hyperlink{class_generic_document_afe94c0abc83a20f2d7dc1ba7677e6238}{ParseStream}<0, \hyperlink{struct_u_t_f16_l_e}{UTF16LE<>} >(eis);  \textcolor{comment}{// 把UTF-16LE文件解析至内存中的UTF-8}

fclose(fp);
\end{DoxyCode}
\hypertarget{md_Cadriciel_Commun_Externe_RapidJSON_doc_stream.zh-cn_EncodedOutputStream}{}\subsection{Encoded\+Output\+Stream}\label{md_Cadriciel_Commun_Externe_RapidJSON_doc_stream.zh-cn_EncodedOutputStream}
{\ttfamily \hyperlink{class_encoded_output_stream}{Encoded\+Output\+Stream}}也是相似的,但它的构造函数有一个{\ttfamily bool put\+B\+OM}参数,用于控制是否在输出字节流写入\+B\+O\+M。


\begin{DoxyCode}
\textcolor{preprocessor}{#include "rapidjson/filewritestream.h"}  \textcolor{comment}{// FileWriteStream}
\textcolor{preprocessor}{#include "rapidjson/encodedstream.h"}    \textcolor{comment}{// EncodedOutputStream}
\textcolor{preprocessor}{#include <cstdio>}

\hyperlink{class_generic_document}{Document} d;         \textcolor{comment}{// Document为GenericDocument<UTF8<> > }
\textcolor{comment}{// ...}

FILE* fp = fopen(\textcolor{stringliteral}{"output\_utf32le.json"}, \textcolor{stringliteral}{"wb"}); \textcolor{comment}{// 非Windows平台使用"w"}

\textcolor{keywordtype}{char} writeBuffer[256];
\hyperlink{class_file_write_stream}{FileWriteStream} bos(fp, writeBuffer, \textcolor{keyword}{sizeof}(writeBuffer));

\textcolor{keyword}{typedef} EncodedOutputStream<UTF32LE<>, \hyperlink{class_file_write_stream}{FileWriteStream}> OutputStream;
OutputStream eos(bos, \textcolor{keyword}{true});   \textcolor{comment}{// 写入BOM}

\hyperlink{class_writer}{Writer<OutputStream, UTF32LE<>}, \hyperlink{struct_u_t_f8}{UTF8<>}> writer(eos);
d.Accept(writer);   \textcolor{comment}{// 这里从内存的UTF-8生成UTF32-LE文件}

fclose(fp);
\end{DoxyCode}
\hypertarget{md_Cadriciel_Commun_Externe_RapidJSON_doc_stream.zh-cn_AutoUTFInputStream}{}\subsection{Auto\+U\+T\+F\+Input\+Stream}\label{md_Cadriciel_Commun_Externe_RapidJSON_doc_stream.zh-cn_AutoUTFInputStream}
有时候,应用软件可能需要㲃理所有可支持的\+J\+S\+O\+N编码。{\ttfamily \hyperlink{class_auto_u_t_f_input_stream}{Auto\+U\+T\+F\+Input\+Stream}}会先使用\+B\+O\+M来检测编码。若\+B\+O\+M不存在,它便会使用合法\+J\+S\+O\+N的特性来检测。若两种方法都失败,它就会倒退至构造函数提供的\+U\+T\+F类型。

由于字符(编码单元/code unit)可能是8位、16位或32位,{\ttfamily \hyperlink{class_auto_u_t_f_input_stream}{Auto\+U\+T\+F\+Input\+Stream}} 需要一个能至少储存32位的字符类型。我们可以使用{\ttfamily unsigned}作为模板参数:


\begin{DoxyCode}
\textcolor{preprocessor}{#include "\hyperlink{document_8h}{rapidjson/document.h}"}
\textcolor{preprocessor}{#include "rapidjson/filereadstream.h"}   \textcolor{comment}{// FileReadStream}
\textcolor{preprocessor}{#include "rapidjson/encodedstream.h"}    \textcolor{comment}{// AutoUTFInputStream}
\textcolor{preprocessor}{#include <cstdio>}

\textcolor{keyword}{using namespace }\hyperlink{namespacerapidjson}{rapidjson};

FILE* fp = fopen(\textcolor{stringliteral}{"any.json"}, \textcolor{stringliteral}{"rb"}); \textcolor{comment}{// 非Windows平台使用"r"}

\textcolor{keywordtype}{char} readBuffer[256];
\hyperlink{class_file_read_stream}{FileReadStream} bis(fp, readBuffer, \textcolor{keyword}{sizeof}(readBuffer));

\hyperlink{class_auto_u_t_f_input_stream}{AutoUTFInputStream<unsigned, FileReadStream>} eis(bis);  \textcolor{comment}{//
       用eis包装bis}

\hyperlink{class_generic_document}{Document} d;         \textcolor{comment}{// Document为GenericDocument<UTF8<> > }
d.\hyperlink{class_generic_document_afe94c0abc83a20f2d7dc1ba7677e6238}{ParseStream}<0, \hyperlink{struct_auto_u_t_f}{AutoUTF<unsigned>} >(eis); \textcolor{comment}{// 把任何UTF编码的文件解析至内存中的UTF-8}

fclose(fp);
\end{DoxyCode}


当要指定流的编码,可使用上面例子中{\ttfamily Parse\+Stream()}的参数{\ttfamily \hyperlink{struct_auto_u_t_f}{Auto\+U\+TF}$<$Char\+Type$>$}。

你可以使用{\ttfamily U\+T\+F\+Type Get\+Type()}去获取\+U\+T\+F类型,并且用{\ttfamily Has\+B\+O\+M()}检测输入流是否含有\+B\+O\+M。\hypertarget{md_Cadriciel_Commun_Externe_RapidJSON_doc_stream.zh-cn_AutoUTFOutputStream}{}\subsection{Auto\+U\+T\+F\+Output\+Stream}\label{md_Cadriciel_Commun_Externe_RapidJSON_doc_stream.zh-cn_AutoUTFOutputStream}
相似地,要在运行时选择输出的编码,我们可使用{\ttfamily \hyperlink{class_auto_u_t_f_output_stream}{Auto\+U\+T\+F\+Output\+Stream}}。这个类本身并非「自动」。你需要在运行时指定\+U\+T\+F类型,以及是否写入\+B\+O\+M。


\begin{DoxyCode}
\textcolor{keyword}{using namespace }\hyperlink{namespacerapidjson}{rapidjson};

\textcolor{keywordtype}{void} WriteJSONFile(FILE* fp, UTFType type, \textcolor{keywordtype}{bool} putBOM, \textcolor{keyword}{const} \hyperlink{class_generic_document}{Document}& d) \{
    \textcolor{keywordtype}{char} writeBuffer[256];
    \hyperlink{class_file_write_stream}{FileWriteStream} bos(fp, writeBuffer, \textcolor{keyword}{sizeof}(writeBuffer));

    \textcolor{keyword}{typedef} \hyperlink{class_auto_u_t_f_output_stream}{AutoUTFOutputStream<unsigned, FileWriteStream>} 
      OutputStream;
    OutputStream eos(bos, type, putBOM);

    \hyperlink{class_writer}{Writer<OutputStream, UTF8<>}, \hyperlink{struct_auto_u_t_f}{AutoUTF<>} > writer;
    d.Accept(writer);
\}
\end{DoxyCode}


{\ttfamily \hyperlink{class_auto_u_t_f_input_stream}{Auto\+U\+T\+F\+Input\+Stream}}/{\ttfamily \hyperlink{class_auto_u_t_f_output_stream}{Auto\+U\+T\+F\+Output\+Stream}}是比{\ttfamily \hyperlink{class_encoded_input_stream}{Encoded\+Input\+Stream}}/{\ttfamily \hyperlink{class_encoded_output_stream}{Encoded\+Output\+Stream}}方便。但前者会产生一点运行期额外开销。\hypertarget{md_Cadriciel_Commun_Externe_RapidJSON_doc_stream.zh-cn_CustomStream}{}\section{Custom Stream}\label{md_Cadriciel_Commun_Externe_RapidJSON_doc_stream.zh-cn_CustomStream}
除了内存/文件流,使用者可创建自行定义适配\+Rapid\+J\+S\+ON A\+P\+I的流类。例如,你可以创建网络流、从压缩文件读取的流等等。

Rapid\+J\+S\+O\+N利用模板结合不同的类型。只要一个类包含所有所需的接口,就可以作为一个流。流的接合定义在{\ttfamily \hyperlink{rapidjson_8h}{rapidjson/rapidjson.\+h}}的注释里:


\begin{DoxyCode}
concept Stream \{
    \textcolor{keyword}{typename} Ch;    

    Ch Peek() \textcolor{keyword}{const};

    Ch Take();

    \textcolor{keywordtype}{size\_t} Tell();

    Ch* PutBegin();

    \textcolor{keywordtype}{void} Put(Ch c);

    \textcolor{keywordtype}{void} Flush();

    \textcolor{keywordtype}{size\_t} PutEnd(Ch* begin);
\}
\end{DoxyCode}


输入流必须实现{\ttfamily Peek()}、{\ttfamily Take()}及{\ttfamily Tell()}。 输出流必须实现{\ttfamily Put()}及{\ttfamily Flush()}。 {\ttfamily Put\+Begin()}及{\ttfamily Put\+End()}是特殊的接口,仅用于原位($\ast$in situ$\ast$)解析。一般的流不需实现它们。然而,即使接口不需用于某些流,仍然需要提供空实现,否则会产生编译错误。\hypertarget{md_Cadriciel_Commun_Externe_RapidJSON_doc_stream.zh-cn_ExampleIStreamWrapper}{}\subsection{Example\+: istream wrapper}\label{md_Cadriciel_Commun_Externe_RapidJSON_doc_stream.zh-cn_ExampleIStreamWrapper}
以下的例子是{\ttfamily std\+::istream}的包装类,它只需现3个函数。


\begin{DoxyCode}
\textcolor{keyword}{class }IStreamWrapper \{
\textcolor{keyword}{public}:
    \textcolor{keyword}{typedef} \textcolor{keywordtype}{char} Ch;

    IStreamWrapper(std::istream& is) : is\_(is) \{
    \}

    Ch Peek()\textcolor{keyword}{ const }\{ \textcolor{comment}{// 1}
        \textcolor{keywordtype}{int} c = is\_.peek();
        \textcolor{keywordflow}{return} c == std::char\_traits<char>::eof() ? \textcolor{charliteral}{'\(\backslash\)0'} : (Ch)c;
    \}

    Ch Take() \{ \textcolor{comment}{// 2}
        \textcolor{keywordtype}{int} c = is\_.get();
        \textcolor{keywordflow}{return} c == std::char\_traits<char>::eof() ? \textcolor{charliteral}{'\(\backslash\)0'} : (Ch)c;
    \}

    \textcolor{keywordtype}{size\_t} Tell()\textcolor{keyword}{ const }\{ \textcolor{keywordflow}{return} (\textcolor{keywordtype}{size\_t})is\_.tellg(); \} \textcolor{comment}{// 3}

    Ch* PutBegin() \{ assert(\textcolor{keyword}{false}); \textcolor{keywordflow}{return} 0; \}
    \textcolor{keywordtype}{void} Put(Ch) \{ assert(\textcolor{keyword}{false}); \}
    \textcolor{keywordtype}{void} Flush() \{ assert(\textcolor{keyword}{false}); \}
    \textcolor{keywordtype}{size\_t} PutEnd(Ch*) \{ assert(\textcolor{keyword}{false}); \textcolor{keywordflow}{return} 0; \}

\textcolor{keyword}{private}:
    IStreamWrapper(\textcolor{keyword}{const} IStreamWrapper&);
    IStreamWrapper& operator=(\textcolor{keyword}{const} IStreamWrapper&);

    std::istream& is\_;
\};
\end{DoxyCode}


使用者能用它来包装{\ttfamily std\+::stringstream}、{\ttfamily std\+::ifstream}的实例。


\begin{DoxyCode}
\textcolor{keyword}{const} \textcolor{keywordtype}{char}* json = \textcolor{stringliteral}{"[1,2,3,4]"};
std::stringstream ss(json);
IStreamWrapper is(ss);

\hyperlink{class_generic_document}{Document} d;
d.\hyperlink{class_generic_document_afe94c0abc83a20f2d7dc1ba7677e6238}{ParseStream}(is);
\end{DoxyCode}


但要注意,由于标准库的内部开销问,此实现的性能可能不如\+Rapid\+J\+S\+O\+N的内存/文件流。\hypertarget{md_Cadriciel_Commun_Externe_RapidJSON_doc_stream.zh-cn_ExampleOStreamWrapper}{}\subsection{Example\+: ostream wrapper}\label{md_Cadriciel_Commun_Externe_RapidJSON_doc_stream.zh-cn_ExampleOStreamWrapper}
以下的例子是{\ttfamily std\+::istream}的包装类,它只需实现2个函数。


\begin{DoxyCode}
\textcolor{keyword}{class }OStreamWrapper \{
\textcolor{keyword}{public}:
    \textcolor{keyword}{typedef} \textcolor{keywordtype}{char} Ch;

    OStreamWrapper(std::ostream& os) : os\_(os) \{
    \}

    Ch Peek()\textcolor{keyword}{ const }\{ assert(\textcolor{keyword}{false}); \textcolor{keywordflow}{return} \textcolor{charliteral}{'\(\backslash\)0'}; \}
    Ch Take() \{ assert(\textcolor{keyword}{false}); \textcolor{keywordflow}{return} \textcolor{charliteral}{'\(\backslash\)0'}; \}
    \textcolor{keywordtype}{size\_t} Tell()\textcolor{keyword}{ const }\{  \}

    Ch* PutBegin() \{ assert(\textcolor{keyword}{false}); \textcolor{keywordflow}{return} 0; \}
    \textcolor{keywordtype}{void} Put(Ch c) \{ os\_.put(c); \}                  \textcolor{comment}{// 1}
    \textcolor{keywordtype}{void} Flush() \{ os\_.flush(); \}                   \textcolor{comment}{// 2}
    \textcolor{keywordtype}{size\_t} PutEnd(Ch*) \{ assert(\textcolor{keyword}{false}); \textcolor{keywordflow}{return} 0; \}

\textcolor{keyword}{private}:
    OStreamWrapper(\textcolor{keyword}{const} OStreamWrapper&);
    OStreamWrapper& operator=(\textcolor{keyword}{const} OStreamWrapper&);

    std::ostream& os\_;
\};
\end{DoxyCode}


使用者能用它来包装{\ttfamily std\+::stringstream}、{\ttfamily std\+::ofstream}的实例。


\begin{DoxyCode}
\hyperlink{class_generic_document}{Document} d;
\textcolor{comment}{// ...}

std::stringstream ss;
OSStreamWrapper os(ss);

\hyperlink{class_writer}{Writer<OStreamWrapper>} writer(os);
d.Accept(writer);
\end{DoxyCode}


但要注意,由于标准库的内部开销问,此实现的性能可能不如\+Rapid\+J\+S\+O\+N的内存/文件流。\hypertarget{md_Cadriciel_Commun_Externe_RapidJSON_doc_stream.zh-cn_Summary}{}\section{Summary}\label{md_Cadriciel_Commun_Externe_RapidJSON_doc_stream.zh-cn_Summary}
本节描述了\+Rapid\+J\+S\+O\+N提供的各种流的类。内存流很简单。若\+J\+S\+O\+N存储在文件中,文件流可减少\+J\+S\+O\+N解析及生成所需的内存量。编码流在字节流和字符流之间作转换。最后,使用者可使用一个简单接口创建自定义的流。 