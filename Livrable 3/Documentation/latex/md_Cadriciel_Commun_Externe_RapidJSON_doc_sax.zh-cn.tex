\char`\"{}\+S\+A\+X\char`\"{}此术语源于\href{http://en.wikipedia.org/wiki/Simple_API_for_XML}{\tt Simple A\+PI for X\+ML}。我们借了此术语去套用在\+J\+S\+O\+N的解析及生成。

在\+Rapid\+J\+S\+O\+N中,{\ttfamily Reader}({\ttfamily \hyperlink{class_generic_reader}{Generic\+Reader}$<$...$>$}的typedef)是\+J\+S\+O\+N的\+S\+A\+X风格解析器,而{\ttfamily \hyperlink{class_writer}{Writer}}({\ttfamily Generic\+Writer$<$...$>$}的typedef)则是\+J\+S\+O\+N的\+S\+A\+X风格生成器。\hypertarget{md_Cadriciel_Commun_Externe_RapidJSON_doc_sax.zh-cn_Reader}{}\section{Reader}\label{md_Cadriciel_Commun_Externe_RapidJSON_doc_sax.zh-cn_Reader}
{\ttfamily Reader}从输入流解析一个\+J\+S\+O\+N。当它从流中读取字符时,它会基于\+J\+S\+O\+N的语法去分析字符,并向处理器发送事件。

例如,以下是一个\+J\+S\+O\+N。


\begin{DoxyCode}
\{
    \textcolor{stringliteral}{"hello"}: \textcolor{stringliteral}{"world"},
    \textcolor{stringliteral}{"t"}: true ,
    \textcolor{stringliteral}{"f"}: \textcolor{keyword}{false},
    \textcolor{stringliteral}{"n"}: null,
    \textcolor{stringliteral}{"i"}: 123,
    \textcolor{stringliteral}{"pi"}: 3.1416,
    \textcolor{stringliteral}{"a"}: [1, 2, 3, 4]
\}
\end{DoxyCode}


当一个{\ttfamily Reader}解析此\+J\+S\+O\+N时,它会顺序地向处理器发送以下的事件:


\begin{DoxyCode}
1 StartObject()
2 Key("hello", 5, true)
3 String("world", 5, true)
4 Key("t", 1, true)
5 Bool(true)
6 Key("f", 1, true)
7 Bool(false)
8 Key("n", 1, true)
9 Null()
10 Key("i")
11 UInt(123)
12 Key("pi")
13 Double(3.1416)
14 Key("a")
15 StartArray()
16 Uint(1)
17 Uint(2)
18 Uint(3)
19 Uint(4)
20 EndArray(4)
21 EndObject(7)
\end{DoxyCode}


除了一些事件参数需要再作解释,这些事件可以轻松地与\+J\+S\+O\+N对上。我们可以看看{\ttfamily simplereader}例子怎样产生和以上完全相同的结果:


\begin{DoxyCode}
\textcolor{preprocessor}{#include "\hyperlink{reader_8h}{rapidjson/reader.h}"}
\textcolor{preprocessor}{#include <iostream>}

\textcolor{keyword}{using namespace }\hyperlink{namespacerapidjson}{rapidjson};
\textcolor{keyword}{using namespace }std;

\textcolor{keyword}{struct }MyHandler \{
    \textcolor{keywordtype}{bool} Null() \{ cout << \textcolor{stringliteral}{"Null()"} << endl; \textcolor{keywordflow}{return} \textcolor{keyword}{true}; \}
    \textcolor{keywordtype}{bool} Bool(\textcolor{keywordtype}{bool} b) \{ cout << \textcolor{stringliteral}{"Bool("} << boolalpha << b << \textcolor{stringliteral}{")"} << endl; \textcolor{keywordflow}{return} \textcolor{keyword}{true}; \}
    \textcolor{keywordtype}{bool} Int(\textcolor{keywordtype}{int} i) \{ cout << \textcolor{stringliteral}{"Int("} << i << \textcolor{stringliteral}{")"} << endl; \textcolor{keywordflow}{return} \textcolor{keyword}{true}; \}
    \textcolor{keywordtype}{bool} Uint(\textcolor{keywordtype}{unsigned} u) \{ cout << \textcolor{stringliteral}{"Uint("} << u << \textcolor{stringliteral}{")"} << endl; \textcolor{keywordflow}{return} \textcolor{keyword}{true}; \}
    \textcolor{keywordtype}{bool} Int64(int64\_t i) \{ cout << \textcolor{stringliteral}{"Int64("} << i << \textcolor{stringliteral}{")"} << endl; \textcolor{keywordflow}{return} \textcolor{keyword}{true}; \}
    \textcolor{keywordtype}{bool} Uint64(uint64\_t u) \{ cout << \textcolor{stringliteral}{"Uint64("} << u << \textcolor{stringliteral}{")"} << endl; \textcolor{keywordflow}{return} \textcolor{keyword}{true}; \}
    \textcolor{keywordtype}{bool} Double(\textcolor{keywordtype}{double} d) \{ cout << \textcolor{stringliteral}{"Double("} << d << \textcolor{stringliteral}{")"} << endl; \textcolor{keywordflow}{return} \textcolor{keyword}{true}; \}
    \textcolor{keywordtype}{bool} String(\textcolor{keyword}{const} \textcolor{keywordtype}{char}* str, \hyperlink{rapidjson_8h_a5ed6e6e67250fadbd041127e6386dcb5}{SizeType} length, \textcolor{keywordtype}{bool} copy) \{ 
        cout << \textcolor{stringliteral}{"String("} << str << \textcolor{stringliteral}{", "} << length << \textcolor{stringliteral}{", "} << boolalpha << copy << \textcolor{stringliteral}{")"} << endl;
        \textcolor{keywordflow}{return} \textcolor{keyword}{true};
    \}
    \textcolor{keywordtype}{bool} StartObject() \{ cout << \textcolor{stringliteral}{"StartObject()"} << endl; \textcolor{keywordflow}{return} \textcolor{keyword}{true}; \}
    \textcolor{keywordtype}{bool} Key(\textcolor{keyword}{const} \textcolor{keywordtype}{char}* str, \hyperlink{rapidjson_8h_a5ed6e6e67250fadbd041127e6386dcb5}{SizeType} length, \textcolor{keywordtype}{bool} copy) \{ 
        cout << \textcolor{stringliteral}{"Key("} << str << \textcolor{stringliteral}{", "} << length << \textcolor{stringliteral}{", "} << boolalpha << copy << \textcolor{stringliteral}{")"} << endl;
        \textcolor{keywordflow}{return} \textcolor{keyword}{true};
    \}
    \textcolor{keywordtype}{bool} EndObject(\hyperlink{rapidjson_8h_a5ed6e6e67250fadbd041127e6386dcb5}{SizeType} memberCount) \{ cout << \textcolor{stringliteral}{"EndObject("} << memberCount << \textcolor{stringliteral}{")"} << endl; \textcolor{keywordflow}{
      return} \textcolor{keyword}{true}; \}
    \textcolor{keywordtype}{bool} StartArray() \{ cout << \textcolor{stringliteral}{"StartArray()"} << endl; \textcolor{keywordflow}{return} \textcolor{keyword}{true}; \}
    \textcolor{keywordtype}{bool} EndArray(\hyperlink{rapidjson_8h_a5ed6e6e67250fadbd041127e6386dcb5}{SizeType} elementCount) \{ cout << \textcolor{stringliteral}{"EndArray("} << elementCount << \textcolor{stringliteral}{")"} << endl; \textcolor{keywordflow}{
      return} \textcolor{keyword}{true}; \}
\};

\textcolor{keywordtype}{void} main() \{
    \textcolor{keyword}{const} \textcolor{keywordtype}{char} json[] = \textcolor{stringliteral}{" \{ \(\backslash\)"hello\(\backslash\)" : \(\backslash\)"world\(\backslash\)", \(\backslash\)"t\(\backslash\)" : true , \(\backslash\)"f\(\backslash\)" : false, \(\backslash\)"n\(\backslash\)": null, \(\backslash\)"i\(\backslash\)":123, \(\backslash\)"
      pi\(\backslash\)": 3.1416, \(\backslash\)"a\(\backslash\)":[1, 2, 3, 4] \} "};

    MyHandler handler;
    \hyperlink{class_generic_reader}{Reader} reader;
    \hyperlink{struct_generic_string_stream}{StringStream} ss(json);
    reader.\hyperlink{class_generic_reader_a0c450620d14ff1824e58bb7bd9b42099}{Parse}(ss, handler);
\}
\end{DoxyCode}


注意\+Rapid\+J\+S\+O\+N使用模板去静态挷定{\ttfamily Reader}类型及处理器的类形,而不是使用含虚函数的类。这个范式可以通过把函数内联而改善性能。\hypertarget{md_Cadriciel_Commun_Externe_RapidJSON_doc_sax.zh-cn_Handler}{}\subsection{Handler}\label{md_Cadriciel_Commun_Externe_RapidJSON_doc_sax.zh-cn_Handler}
如前例所示,使用者需要实现一个处理器(handler),用于处理来自{\ttfamily Reader}的事件(函数调用)。处理器必须包含以下的成员函数。


\begin{DoxyCode}
\textcolor{keyword}{class }Handler \{
    \textcolor{keywordtype}{bool} Null();
    \textcolor{keywordtype}{bool} Bool(\textcolor{keywordtype}{bool} b);
    \textcolor{keywordtype}{bool} Int(\textcolor{keywordtype}{int} i);
    \textcolor{keywordtype}{bool} Uint(\textcolor{keywordtype}{unsigned} i);
    \textcolor{keywordtype}{bool} Int64(int64\_t i);
    \textcolor{keywordtype}{bool} Uint64(uint64\_t i);
    \textcolor{keywordtype}{bool} Double(\textcolor{keywordtype}{double} d);
    \textcolor{keywordtype}{bool} String(\textcolor{keyword}{const} Ch* str, \hyperlink{rapidjson_8h_a5ed6e6e67250fadbd041127e6386dcb5}{SizeType} length, \textcolor{keywordtype}{bool} copy);
    \textcolor{keywordtype}{bool} StartObject();
    \textcolor{keywordtype}{bool} Key(\textcolor{keyword}{const} Ch* str, \hyperlink{rapidjson_8h_a5ed6e6e67250fadbd041127e6386dcb5}{SizeType} length, \textcolor{keywordtype}{bool} copy);
    \textcolor{keywordtype}{bool} EndObject(\hyperlink{rapidjson_8h_a5ed6e6e67250fadbd041127e6386dcb5}{SizeType} memberCount);
    \textcolor{keywordtype}{bool} StartArray();
    \textcolor{keywordtype}{bool} EndArray(\hyperlink{rapidjson_8h_a5ed6e6e67250fadbd041127e6386dcb5}{SizeType} elementCount);
\};
\end{DoxyCode}


当{\ttfamily Reader}遇到\+J\+S\+ON null值时会调用{\ttfamily Null()}。

当{\ttfamily Reader}遇到\+J\+S\+ON true或false值时会调用{\ttfamily Bool(bool)}。

当{\ttfamily Reader}遇到\+J\+S\+ON number,它会选择一个合适的\+C++类型映射,然后调用{\ttfamily Int(int)}、{\ttfamily Uint(unsigned)}、{\ttfamily Int64(int64\+\_\+t)}、{\ttfamily Uint64(uint64\+\_\+t)}及{\ttfamily Double(double)}的$\ast$其中之一个$\ast$。

当{\ttfamily Reader}遇到\+J\+S\+ON string,它会调用{\ttfamily String(const char$\ast$ str, Size\+Type length, bool copy)}。第一个参数是字符串的指针。第二个参数是字符串的长度(不包含空终止符号)。注意\+Rapid\+J\+S\+O\+N支持字串中含有空字符`\textquotesingle{}\textbackslash{}0\textquotesingle{}{\ttfamily 。若出现这种情况,便会有}strlen(str) $<$ length{\ttfamily 。最后的}copy{\ttfamily 参数表示处理器是否需要复制该字符串。在正常解析时,}copy = true{\ttfamily 。仅当使用原位解析时,}copy = false`。此外,还要注意字符的类型与目标编码相关,我们稍后会再谈这一点。

当{\ttfamily Reader}遇到\+J\+S\+ON object的开始之时,它会调用{\ttfamily Start\+Object()}。\+J\+S\+O\+N的object是一个键值对(成员)的集合。若object包含成员,它会先为成员的名字调用{\ttfamily Key()},然后再按值的类型调用函数。它不断调用这些键值对,直至最终调用{\ttfamily End\+Object(\+Size\+Type member\+Count)}。注意{\ttfamily member\+Count}参数对处理器来说只是协助性质,使用者可能不需要此参数。

J\+S\+ON array与object相似,但更简单。在array开始时,{\ttfamily Reader}会调用{\ttfamily Begin\+Arary()}。若array含有元素,它会按元素的类型来读用函数。相似地,最后它会调用{\ttfamily End\+Array(\+Size\+Type element\+Count)},其中{\ttfamily element\+Count}参数对处理器来说只是协助性质。

每个处理器函数都返回一个{\ttfamily bool}。正常它们应返回{\ttfamily true}。若处理器遇到错误,它可以返回{\ttfamily false}去通知事件发送方停止继续处理。

例如,当我们用{\ttfamily Reader}解析一个\+J\+S\+O\+N时,处理器检测到该\+J\+S\+O\+N并不符合所需的schema,那么处理器可以返回{\ttfamily false},令{\ttfamily Reader}停止之后的解析工作。而{\ttfamily Reader}会进入一个错误状态,并以{\ttfamily k\+Parse\+Error\+Termination}错误码标识。\hypertarget{md_Cadriciel_Commun_Externe_RapidJSON_doc_sax.zh-cn_GenericReader}{}\subsection{Generic\+Reader}\label{md_Cadriciel_Commun_Externe_RapidJSON_doc_sax.zh-cn_GenericReader}
前面提及,{\ttfamily Reader}是{\ttfamily \hyperlink{class_generic_reader}{Generic\+Reader}}模板类的typedef:


\begin{DoxyCode}
\textcolor{keyword}{namespace }\hyperlink{namespacerapidjson}{rapidjson} \{

\textcolor{keyword}{template} <\textcolor{keyword}{typename} SourceEncoding, \textcolor{keyword}{typename} TargetEncoding, \textcolor{keyword}{typename} Allocator = MemoryPoolAllocator<> >
\textcolor{keyword}{class }\hyperlink{class_generic_reader}{GenericReader} \{
    \textcolor{comment}{// ...}
\};

\textcolor{keyword}{typedef} \hyperlink{class_generic_reader}{GenericReader<UTF8<>}, \hyperlink{struct_u_t_f8}{UTF8<>} > \hyperlink{reader_8h_a84f3b66a66647f4ac4267078359188ba}{Reader};

\} \textcolor{comment}{// namespace rapidjson}
\end{DoxyCode}


{\ttfamily Reader}使用\+U\+T\+F-\/8作为来源及目标编码。来源编码是指\+J\+S\+O\+N流的编码。目标编码是指{\ttfamily String()}的{\ttfamily str}参数所用的编码。例如,要解析一个\+U\+T\+F-\/8流并输出至\+U\+T\+F-\/16 string事件,你需要这么定义一个reader:


\begin{DoxyCode}
\hyperlink{class_generic_reader}{GenericReader<UTF8<>}, \hyperlink{struct_u_t_f16}{UTF16<>} > reader;
\end{DoxyCode}


注意到{\ttfamily \hyperlink{struct_u_t_f16}{U\+T\+F16}}的缺省类型是{\ttfamily wchar\+\_\+t}。因此这个{\ttfamily reader}需要调用处理器的{\ttfamily String(const wchar\+\_\+t$\ast$, Size\+Type, bool)}。

第三个模板参数{\ttfamily Allocator}是内部数据结构(实际上是一个堆栈)的分配器类型。\hypertarget{md_Cadriciel_Commun_Externe_RapidJSON_doc_sax.zh-cn_Parsing}{}\subsection{Parsing}\label{md_Cadriciel_Commun_Externe_RapidJSON_doc_sax.zh-cn_Parsing}
{\ttfamily Reader}的唯一功能就是解析\+J\+S\+O\+N。


\begin{DoxyCode}
\textcolor{keyword}{template} <\textcolor{keywordtype}{unsigned} parseFlags, \textcolor{keyword}{typename} InputStream, \textcolor{keyword}{typename} Handler>
\textcolor{keywordtype}{bool} Parse(InputStream& is, Handler& handler);

\textcolor{comment}{// 使用 parseFlags = kDefaultParseFlags}
\textcolor{keyword}{template} <\textcolor{keyword}{typename} InputStream, \textcolor{keyword}{typename} Handler>
\textcolor{keywordtype}{bool} Parse(InputStream& is, Handler& handler);
\end{DoxyCode}


若在解析中出现错误,它会返回{\ttfamily false}。使用者可调用{\ttfamily bool Has\+Parse\+Eror()}, {\ttfamily Parse\+Error\+Code Get\+Parse\+Error\+Code()}及{\ttfamily size\+\_\+t Get\+Error\+Offset()}获取错误状态。实际上{\ttfamily Document}使用这些{\ttfamily Reader}函数去获取解析错误。请参考D\+OM去了解有关解析错误的细节。\hypertarget{md_Cadriciel_Commun_Externe_RapidJSON_doc_sax.zh-cn_Writer}{}\section{Writer}\label{md_Cadriciel_Commun_Externe_RapidJSON_doc_sax.zh-cn_Writer}
{\ttfamily Reader}把\+J\+S\+O\+N转换(解析)成为事件。{\ttfamily \hyperlink{class_writer}{Writer}}完全做相反的事情。它把事件转换成\+J\+S\+O\+N。

{\ttfamily \hyperlink{class_writer}{Writer}}是非常容易使用的。若你的应用程序只需把一些数据转换成\+J\+S\+O\+N,可能直接使用{\ttfamily \hyperlink{class_writer}{Writer}},会比建立一个{\ttfamily Document}然后用{\ttfamily \hyperlink{class_writer}{Writer}}把它转换成\+J\+S\+O\+N更加方便。

在{\ttfamily simplewriter}例子里,我们做{\ttfamily simplereader}完全相反的事情。


\begin{DoxyCode}
\textcolor{preprocessor}{#include "rapidjson/writer.h"}
\textcolor{preprocessor}{#include "rapidjson/stringbuffer.h"}
\textcolor{preprocessor}{#include <iostream>}

\textcolor{keyword}{using namespace }\hyperlink{namespacerapidjson}{rapidjson};
\textcolor{keyword}{using namespace }std;

\textcolor{keywordtype}{void} main() \{
    \hyperlink{class_generic_string_buffer}{StringBuffer} s;
    \hyperlink{class_writer}{Writer<StringBuffer>} writer(s);

    writer.StartObject();
    writer.Key(\textcolor{stringliteral}{"hello"});
    writer.String(\textcolor{stringliteral}{"world"});
    writer.Key(\textcolor{stringliteral}{"t"});
    writer.Bool(\textcolor{keyword}{true});
    writer.Key(\textcolor{stringliteral}{"f"});
    writer.Bool(\textcolor{keyword}{false});
    writer.Key(\textcolor{stringliteral}{"n"});
    writer.Null();
    writer.Key(\textcolor{stringliteral}{"i"});
    writer.Uint(123);
    writer.Key(\textcolor{stringliteral}{"pi"});
    writer.Double(3.1416);
    writer.Key(\textcolor{stringliteral}{"a"});
    writer.StartArray();
    \textcolor{keywordflow}{for} (\textcolor{keywordtype}{unsigned} i = 0; i < 4; i++)
        writer.Uint(i);
    writer.EndArray();
    writer.EndObject();

    cout << s.GetString() << endl;
\}
\end{DoxyCode}



\begin{DoxyCode}
1 \{"hello":"world","t":true,"f":false,"n":null,"i":123,"pi":3.1416,"a":[0,1,2,3]\}
\end{DoxyCode}


{\ttfamily String()}及{\ttfamily Key()}各有两个重载。一个是如处理器concept般,有3个参数。它能处理含空字符的字符串。另一个是如上中使用的较简单版本。

注意到,例子代码中的{\ttfamily End\+Array()}及{\ttfamily End\+Object()}并没有参数。可以传递一个{\ttfamily Size\+Type}的参数,但它会被{\ttfamily \hyperlink{class_writer}{Writer}}忽略。

你可能会怀疑,为什么不使用{\ttfamily sprintf()}或{\ttfamily std\+::stringstream}去建立一个\+J\+S\+O\+N?

这有几个原因:
\begin{DoxyEnumerate}
\item {\ttfamily \hyperlink{class_writer}{Writer}}必然会输出一个结构良好(well-\/formed)的\+J\+S\+O\+N。若然有错误的事件次序(如{\ttfamily Int()}紧随{\ttfamily Start\+Object()}出现),它会在调试模式中产生断言失败。
\item {\ttfamily Writer\+::\+String()}可处理字符串转义(如把码点{\ttfamily U+000A}转换成{\ttfamily \textbackslash{}n})及进行\+Unicode转码。
\item {\ttfamily \hyperlink{class_writer}{Writer}}一致地处理number的输出。
\item {\ttfamily \hyperlink{class_writer}{Writer}}实现了事件处理器concept。可用于处理来自{\ttfamily Reader}、{\ttfamily Document}或其他事件发生器。
\item {\ttfamily \hyperlink{class_writer}{Writer}}可对不同平台进行优化。
\end{DoxyEnumerate}

无论如何,使用{\ttfamily \hyperlink{class_writer}{Writer}} A\+P\+I去生成\+J\+S\+O\+N甚至乎比这些临时方法更简单。\hypertarget{md_Cadriciel_Commun_Externe_RapidJSON_doc_sax.zh-cn_WriterTemplate}{}\subsection{Template}\label{md_Cadriciel_Commun_Externe_RapidJSON_doc_sax.zh-cn_WriterTemplate}
{\ttfamily \hyperlink{class_writer}{Writer}}与{\ttfamily Reader}有少许设计区别。{\ttfamily \hyperlink{class_writer}{Writer}}是一个模板类,而不是一个typedef。 并没有{\ttfamily Generic\+Writer}。以下是{\ttfamily \hyperlink{class_writer}{Writer}}的声明。


\begin{DoxyCode}
\textcolor{keyword}{namespace }\hyperlink{namespacerapidjson}{rapidjson} \{

\textcolor{keyword}{template}<\textcolor{keyword}{typename} OutputStream, \textcolor{keyword}{typename} SourceEncoding = UTF8<>, \textcolor{keyword}{typename} TargetEncoding = UTF8<>, \textcolor{keyword}{
      typename} Allocator = CrtAllocator<> >
\textcolor{keyword}{class }\hyperlink{class_writer}{Writer} \{
\textcolor{keyword}{public}:
    \hyperlink{class_writer}{Writer}(OutputStream& os, Allocator* allocator = 0, \textcolor{keywordtype}{size\_t} levelDepth = kDefaultLevelDepth)
\textcolor{comment}{// ...}
\};

\} \textcolor{comment}{// namespace rapidjson}
\end{DoxyCode}


{\ttfamily Output\+Stream}模板参数是输出流的类型。它的类型不可以被自动推断,必须由使用者提供。

{\ttfamily Source\+Encoding}模板参数指定了{\ttfamily String(const Ch$\ast$, ...)}的编码。

{\ttfamily Target\+Encoding}模板参数指定输出流的编码。

最后一个{\ttfamily Allocator}是分配器的类型,用于分配内部数据结构(一个堆栈)。

此外,{\ttfamily \hyperlink{class_writer}{Writer}}的构造函数有一{\ttfamily level\+Depth}参数。存储每层阶信息的初始内存分配量受此参数影响。\hypertarget{md_Cadriciel_Commun_Externe_RapidJSON_doc_sax.zh-cn_PrettyWriter}{}\subsection{Pretty\+Writer}\label{md_Cadriciel_Commun_Externe_RapidJSON_doc_sax.zh-cn_PrettyWriter}
{\ttfamily \hyperlink{class_writer}{Writer}}所输出的是没有空格字符的最紧凑\+J\+S\+O\+N,适合网络传输或储存,但就适合人类阅读。

因此,\+Rapid\+J\+S\+O\+N提供了一个{\ttfamily \hyperlink{class_pretty_writer}{Pretty\+Writer}},它在输出中加入缩进及换行。

{\ttfamily \hyperlink{class_pretty_writer}{Pretty\+Writer}}的用法与{\ttfamily \hyperlink{class_writer}{Writer}}几乎一样,不同之处是{\ttfamily \hyperlink{class_pretty_writer}{Pretty\+Writer}}提供了一个{\ttfamily Set\+Indent(\+Ch indent\+Char, unsigned indent\+Char\+Count)}函数。缺省的缩进是4个空格。\hypertarget{md_Cadriciel_Commun_Externe_RapidJSON_doc_sax.zh-cn_CompletenessReset}{}\subsection{Completeness and Reset}\label{md_Cadriciel_Commun_Externe_RapidJSON_doc_sax.zh-cn_CompletenessReset}
一个{\ttfamily \hyperlink{class_writer}{Writer}}只可输出单个\+J\+S\+O\+N,其根节点可以是任何\+J\+S\+O\+N类型。当处理完单个根节点事件(如{\ttfamily String()}),或匹配的最后{\ttfamily End\+Object()}或{\ttfamily End\+Array()}事件,输出的\+J\+S\+O\+N是结构完整(well-\/formed)及完整的。使用者可调用{\ttfamily \hyperlink{class_writer_aced42429d1b31a565c5ca0310bf4e276}{Writer\+::\+Is\+Complete()}}去检测完整性。

当\+J\+S\+O\+N完整时,{\ttfamily \hyperlink{class_writer}{Writer}}不能再接受新的事件。不然其输出便会是不合法的(例如有超过一个根节点)。为了重新利用{\ttfamily \hyperlink{class_writer}{Writer}}对象,使用者可调用{\ttfamily \hyperlink{class_writer_a4e5bd5e6364edca476125b511b3dca9c}{Writer\+::\+Reset(\+Output\+Stream\& os)}}去重置其所有内部状态及设置新的输出流。\hypertarget{md_Cadriciel_Commun_Externe_RapidJSON_doc_sax.zh-cn_Techniques}{}\section{Techniques}\label{md_Cadriciel_Commun_Externe_RapidJSON_doc_sax.zh-cn_Techniques}
\hypertarget{md_Cadriciel_Commun_Externe_RapidJSON_doc_sax.zh-cn_CustomDataStructure}{}\subsection{Parsing J\+S\+O\+N to Custom Data Structure}\label{md_Cadriciel_Commun_Externe_RapidJSON_doc_sax.zh-cn_CustomDataStructure}
{\ttfamily Document}的解析功能完全依靠{\ttfamily Reader}。实际上{\ttfamily Document}是一个处理器,在解析\+J\+S\+O\+N时接收事件去建立一个\+D\+O\+M。

使用者可以直接使用{\ttfamily Reader}去建立其他数据结构。这消除了建立\+D\+O\+M的步骤,从而减少了内存开销并改善性能。

在以下的{\ttfamily messagereader}例子中,{\ttfamily Parse\+Messages()}解析一个\+J\+S\+O\+N,该\+J\+S\+O\+N应该是一个含键值对的object。


\begin{DoxyCode}
\textcolor{preprocessor}{#include "\hyperlink{reader_8h}{rapidjson/reader.h}"}
\textcolor{preprocessor}{#include "rapidjson/error/en.h"}
\textcolor{preprocessor}{#include <iostream>}
\textcolor{preprocessor}{#include <string>}
\textcolor{preprocessor}{#include <map>}

\textcolor{keyword}{using namespace }std;
\textcolor{keyword}{using namespace }\hyperlink{namespacerapidjson}{rapidjson};

\textcolor{keyword}{typedef} map<string, string> MessageMap;

\textcolor{keyword}{struct }MessageHandler
    : \textcolor{keyword}{public} \hyperlink{struct_base_reader_handler}{BaseReaderHandler}<UTF8<>, MessageHandler> \{
    MessageHandler() : state\_(kExpectObjectStart) \{
    \}

    \textcolor{keywordtype}{bool} StartObject() \{
        \textcolor{keywordflow}{switch} (state\_) \{
        \textcolor{keywordflow}{case} kExpectObjectStart:
            state\_ = kExpectNameOrObjectEnd;
            \textcolor{keywordflow}{return} \textcolor{keyword}{true};
        \textcolor{keywordflow}{default}:
            \textcolor{keywordflow}{return} \textcolor{keyword}{false};
        \}
    \}

    \textcolor{keywordtype}{bool} String(\textcolor{keyword}{const} \textcolor{keywordtype}{char}* str, \hyperlink{rapidjson_8h_a5ed6e6e67250fadbd041127e6386dcb5}{SizeType} length, \textcolor{keywordtype}{bool}) \{
        \textcolor{keywordflow}{switch} (state\_) \{
        \textcolor{keywordflow}{case} kExpectNameOrObjectEnd:
            name\_ = string(str, length);
            state\_ = kExpectValue;
            \textcolor{keywordflow}{return} \textcolor{keyword}{true};
        \textcolor{keywordflow}{case} kExpectValue:
            messages\_.insert(MessageMap::value\_type(name\_, \textcolor{keywordtype}{string}(str, length)));
            state\_ = kExpectNameOrObjectEnd;
            \textcolor{keywordflow}{return} \textcolor{keyword}{true};
        \textcolor{keywordflow}{default}:
            \textcolor{keywordflow}{return} \textcolor{keyword}{false};
        \}
    \}

    \textcolor{keywordtype}{bool} EndObject(\hyperlink{rapidjson_8h_a5ed6e6e67250fadbd041127e6386dcb5}{SizeType}) \{ \textcolor{keywordflow}{return} state\_ == kExpectNameOrObjectEnd; \}

    \textcolor{keywordtype}{bool} Default() \{ \textcolor{keywordflow}{return} \textcolor{keyword}{false}; \} \textcolor{comment}{// All other events are invalid.}

    MessageMap messages\_;
    \textcolor{keyword}{enum} State \{
        kExpectObjectStart,
        kExpectNameOrObjectEnd,
        kExpectValue,
    \}state\_;
    std::string name\_;
\};

\textcolor{keywordtype}{void} ParseMessages(\textcolor{keyword}{const} \textcolor{keywordtype}{char}* json, MessageMap& messages) \{
    \hyperlink{class_generic_reader}{Reader} reader;
    MessageHandler handler;
    \hyperlink{struct_generic_string_stream}{StringStream} ss(json);
    \textcolor{keywordflow}{if} (reader.\hyperlink{class_generic_reader_a0c450620d14ff1824e58bb7bd9b42099}{Parse}(ss, handler))
        messages.swap(handler.messages\_);   \textcolor{comment}{// Only change it if success.}
    \textcolor{keywordflow}{else} \{
        \hyperlink{group___r_a_p_i_d_j_s_o_n___e_r_r_o_r_s_ga8d4b32dfc45840bca189ade2bbcb6ba7}{ParseErrorCode} e = reader.\hyperlink{class_generic_reader_ac45a26246877c4daa85021ae67caa017}{GetParseErrorCode}();
        \textcolor{keywordtype}{size\_t} o = reader.\hyperlink{class_generic_reader_a77399ac40cca1fb113a2d507f476b4e7}{GetErrorOffset}();
        cout << \textcolor{stringliteral}{"Error: "} << \hyperlink{group___r_a_p_i_d_j_s_o_n___e_r_r_o_r_s_ga755b523205f46c980c80d12e230a3abd}{GetParseError\_En}(e) << endl;;
        cout << \textcolor{stringliteral}{" at offset "} << o << \textcolor{stringliteral}{" near '"} << string(json).substr(o, 10) << \textcolor{stringliteral}{"...'"} << endl;
    \}
\}

\textcolor{keywordtype}{int} main() \{
    MessageMap messages;

    \textcolor{keyword}{const} \textcolor{keywordtype}{char}* json1 = \textcolor{stringliteral}{"\{ \(\backslash\)"greeting\(\backslash\)" : \(\backslash\)"Hello!\(\backslash\)", \(\backslash\)"farewell\(\backslash\)" : \(\backslash\)"bye-bye!\(\backslash\)" \}"};
    cout << json1 << endl;
    ParseMessages(json1, messages);

    \textcolor{keywordflow}{for} (MessageMap::const\_iterator itr = messages.begin(); itr != messages.end(); ++itr)
        cout << itr->first << \textcolor{stringliteral}{": "} << itr->second << endl;

    cout << endl << \textcolor{stringliteral}{"Parse a JSON with invalid schema."} << endl;
    \textcolor{keyword}{const} \textcolor{keywordtype}{char}* json2 = \textcolor{stringliteral}{"\{ \(\backslash\)"greeting\(\backslash\)" : \(\backslash\)"Hello!\(\backslash\)", \(\backslash\)"farewell\(\backslash\)" : \(\backslash\)"bye-bye!\(\backslash\)", \(\backslash\)"foo\(\backslash\)" : \{\} \}"};
    cout << json2 << endl;
    ParseMessages(json2, messages);

    \textcolor{keywordflow}{return} 0;
\}
\end{DoxyCode}



\begin{DoxyCode}
1 \{ "greeting" : "Hello!", "farewell" : "bye-bye!" \}
2 farewell: bye-bye!
3 greeting: Hello!
4 
5 Parse a JSON with invalid schema.
6 \{ "greeting" : "Hello!", "farewell" : "bye-bye!", "foo" : \{\} \}
7 Error: Terminate parsing due to Handler error.
8  at offset 59 near '\} \}...'
\end{DoxyCode}


第一个\+J\+S\+O\+N({\ttfamily json1})被成功地解析至{\ttfamily Message\+Map}。由于{\ttfamily Message\+Map}是一个{\ttfamily std\+::map},列印次序按键值排序。此次序与\+J\+S\+O\+N中的次序不同。

在第二个\+J\+S\+O\+N({\ttfamily json2})中,{\ttfamily foo}的值是一个空object。由于它是一个object,{\ttfamily Message\+Handler\+::\+Start\+Object()}会被调用。然而,在{\ttfamily state\+\_\+ = k\+Expect\+Value}的情况下,该函数会返回{\ttfamily false},并令到解析过程终止。错误代码是{\ttfamily k\+Parse\+Error\+Termination}。\hypertarget{md_Cadriciel_Commun_Externe_RapidJSON_doc_sax.zh-cn_Filtering}{}\subsection{Filtering of J\+S\+ON}\label{md_Cadriciel_Commun_Externe_RapidJSON_doc_sax.zh-cn_Filtering}
如前面提及过,{\ttfamily \hyperlink{class_writer}{Writer}}可处理{\ttfamily Reader}发出的事件。{\ttfamily condense}例子简单地设置{\ttfamily \hyperlink{class_writer}{Writer}}作为一个{\ttfamily Reader}的处理器,因此它能移除\+J\+S\+O\+N中的所有空白字符。{\ttfamily pretty}例子使用同样的关系,只是以{\ttfamily \hyperlink{class_pretty_writer}{Pretty\+Writer}}取代{\ttfamily \hyperlink{class_writer}{Writer}}。因此{\ttfamily pretty}能够重新格式化\+J\+S\+O\+N,加入缩进及换行。

实际上,我们可以使用\+S\+A\+X风格\+A\+P\+I去加入(多个)中间层去过滤\+J\+S\+O\+N的内容。例如{\ttfamily capitalize}例子可以把所有\+J\+S\+ON string改为大写。


\begin{DoxyCode}
\textcolor{preprocessor}{#include "\hyperlink{reader_8h}{rapidjson/reader.h}"}
\textcolor{preprocessor}{#include "rapidjson/writer.h"}
\textcolor{preprocessor}{#include "rapidjson/filereadstream.h"}
\textcolor{preprocessor}{#include "rapidjson/filewritestream.h"}
\textcolor{preprocessor}{#include "rapidjson/error/en.h"}
\textcolor{preprocessor}{#include <vector>}
\textcolor{preprocessor}{#include <cctype>}

\textcolor{keyword}{using namespace }\hyperlink{namespacerapidjson}{rapidjson};

\textcolor{keyword}{template}<\textcolor{keyword}{typename} OutputHandler>
\textcolor{keyword}{struct }CapitalizeFilter \{
    CapitalizeFilter(OutputHandler& out) : out\_(out), buffer\_() \{
    \}

    \textcolor{keywordtype}{bool} Null() \{ \textcolor{keywordflow}{return} out\_.Null(); \}
    \textcolor{keywordtype}{bool} Bool(\textcolor{keywordtype}{bool} b) \{ \textcolor{keywordflow}{return} out\_.Bool(b); \}
    \textcolor{keywordtype}{bool} Int(\textcolor{keywordtype}{int} i) \{ \textcolor{keywordflow}{return} out\_.Int(i); \}
    \textcolor{keywordtype}{bool} Uint(\textcolor{keywordtype}{unsigned} u) \{ \textcolor{keywordflow}{return} out\_.Uint(u); \}
    \textcolor{keywordtype}{bool} Int64(int64\_t i) \{ \textcolor{keywordflow}{return} out\_.Int64(i); \}
    \textcolor{keywordtype}{bool} Uint64(uint64\_t u) \{ \textcolor{keywordflow}{return} out\_.Uint64(u); \}
    \textcolor{keywordtype}{bool} Double(\textcolor{keywordtype}{double} d) \{ \textcolor{keywordflow}{return} out\_.Double(d); \}
    \textcolor{keywordtype}{bool} String(\textcolor{keyword}{const} \textcolor{keywordtype}{char}* str, \hyperlink{rapidjson_8h_a5ed6e6e67250fadbd041127e6386dcb5}{SizeType} length, \textcolor{keywordtype}{bool}) \{ 
        buffer\_.clear();
        \textcolor{keywordflow}{for} (\hyperlink{rapidjson_8h_a5ed6e6e67250fadbd041127e6386dcb5}{SizeType} i = 0; i < \hyperlink{group__core__func__geometric_ga03b2831439defb8922832b540b91b8a7}{length}; i++)
            buffer\_.push\_back(std::toupper(str[i]));
        \textcolor{keywordflow}{return} out\_.String(&buffer\_.front(), \hyperlink{group__core__func__geometric_ga03b2831439defb8922832b540b91b8a7}{length}, \textcolor{keyword}{true}); \textcolor{comment}{// true = output handler need to copy the
       string}
    \}
    \textcolor{keywordtype}{bool} StartObject() \{ \textcolor{keywordflow}{return} out\_.StartObject(); \}
    \textcolor{keywordtype}{bool} Key(\textcolor{keyword}{const} \textcolor{keywordtype}{char}* str, \hyperlink{rapidjson_8h_a5ed6e6e67250fadbd041127e6386dcb5}{SizeType} length, \textcolor{keywordtype}{bool} copy) \{ \textcolor{keywordflow}{return} String(str, length, copy); \}
    \textcolor{keywordtype}{bool} EndObject(\hyperlink{rapidjson_8h_a5ed6e6e67250fadbd041127e6386dcb5}{SizeType} memberCount) \{ \textcolor{keywordflow}{return} out\_.EndObject(memberCount); \}
    \textcolor{keywordtype}{bool} StartArray() \{ \textcolor{keywordflow}{return} out\_.StartArray(); \}
    \textcolor{keywordtype}{bool} EndArray(\hyperlink{rapidjson_8h_a5ed6e6e67250fadbd041127e6386dcb5}{SizeType} elementCount) \{ \textcolor{keywordflow}{return} out\_.EndArray(elementCount); \}

    OutputHandler& out\_;
    std::vector<char> buffer\_;
\};

\textcolor{keywordtype}{int} main(\textcolor{keywordtype}{int}, \textcolor{keywordtype}{char}*[]) \{
    \textcolor{comment}{// Prepare JSON reader and input stream.}
    \hyperlink{class_generic_reader}{Reader} reader;
    \textcolor{keywordtype}{char} readBuffer[65536];
    \hyperlink{class_file_read_stream}{FileReadStream} is(stdin, readBuffer, \textcolor{keyword}{sizeof}(readBuffer));

    \textcolor{comment}{// Prepare JSON writer and output stream.}
    \textcolor{keywordtype}{char} writeBuffer[65536];
    \hyperlink{class_file_write_stream}{FileWriteStream} os(stdout, writeBuffer, \textcolor{keyword}{sizeof}(writeBuffer));
    \hyperlink{class_writer}{Writer<FileWriteStream>} writer(os);

    \textcolor{comment}{// JSON reader parse from the input stream and let writer generate the output.}
    CapitalizeFilter<Writer<FileWriteStream> > filter(writer);
    \textcolor{keywordflow}{if} (!reader.\hyperlink{class_generic_reader_a0c450620d14ff1824e58bb7bd9b42099}{Parse}(is, filter)) \{
        fprintf(stderr, \textcolor{stringliteral}{"\(\backslash\)nError(%u): %s\(\backslash\)n"}, (\textcolor{keywordtype}{unsigned})reader.\hyperlink{class_generic_reader_a77399ac40cca1fb113a2d507f476b4e7}{GetErrorOffset}(), 
      \hyperlink{group___r_a_p_i_d_j_s_o_n___e_r_r_o_r_s_ga755b523205f46c980c80d12e230a3abd}{GetParseError\_En}(reader.\hyperlink{class_generic_reader_ac45a26246877c4daa85021ae67caa017}{GetParseErrorCode}()));
        \textcolor{keywordflow}{return} 1;
    \}

    \textcolor{keywordflow}{return} 0;
\}
\end{DoxyCode}


注意到,不可简单地把\+J\+S\+O\+N当作字符串去改为大写。例如: 
\begin{DoxyCode}
1 ["Hello\(\backslash\)nWorld"]
\end{DoxyCode}


简单地把整个\+J\+S\+O\+N转为大写的话会产生错误的转义符: 
\begin{DoxyCode}
1 ["HELLO\(\backslash\)NWORLD"]
\end{DoxyCode}


而{\ttfamily capitalize}就会产生正确的结果: 
\begin{DoxyCode}
1 ["HELLO\(\backslash\)nWORLD"]
\end{DoxyCode}


我们还可以开发更复杂的过滤器。然而,由于\+S\+A\+X风格\+A\+P\+I在某一时间点只能提供单一事件的信息,使用者需要自行记录一些上下文信息(例如从根节点起的路径、储存其他相关值)。对些一些处理情况,用\+D\+O\+M会比\+S\+A\+X更容易实现。 